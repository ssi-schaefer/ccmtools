
%==============================================================================
\chapter{Process Overview}
%==============================================================================

The CCM Tools currently concentrate on C++ component development. Use the
"ccmtools-c++-{generate|make|install|uninstall}" scripts to drive the Java
classes. Double check that you have both Python ($\ge 2.1$) and Confix ($\ge
1.1$) installed before beginning.

The basic development process follows a few simple steps. These are described in
further detail later in this chapter.

\begin{enumerate}
\item Change directories to your component development directory.
\item Edit/create all source IDL files required for your components.
\item Set the CCMTOOLS\_PACKAGE environment variable to the name of your
      component set.
\item Run {\tt ccmtools-c++-generate [options] *.idl} to generate component code
      and skeleton application logic files. Use {\tt ccmtools-c++-generate
      -\-help} to get a list of available options. (All of the CCM Tools C++
      scripts support the GNU standard options of {\tt -\-help} and {\tt
      -\-version}, along with other script-specific options.)
\item Edit the *\_app.cc and *\_user\_types.h files.
\item Run {\tt ccmtools-c++-make} to compile and test the components.
\item Run {\tt ccmtools-c++-install} to install the components.
\item If you need to later, run {\tt ccmtools-c++-uninstall} to uninstall the
      components.
\end{enumerate}

If you want to use the CCM Tools for another code generation purpose, use the
more generic "ccmtools-generate" script, described in
Chapter~\ref{chapter:ccmtools-generate}.

