%==============================================================================
\chapter{Local CCM Components}
%==============================================================================

From practical experiences we have learned that there is a need for improvements in
CCM to be usable. The following sections describe the new concepts 
\cite{TMKKW-EUROMICRO:2002,TKKKW-MIDDLEWARE:2003}
we introduced in respect to the CCM specification.


%==============================================================================
\section{Remote components}
%==============================================================================

The CCM specification describes only remote components where all ports are
accessible from CORBA clients. Some open source implementations like MicoCCM
\cite{MicoCCM} and OpenCCM \cite{MarvieMerle2001} implement this type of
components.

In such implementations, each remote component is accessible from any point in
the network, but communication between components in the same address space is
very expensive because method calls still have to go through the CORBA Object
Request Broker (ORB).

There are some techniques for transparently optimizing communication overhead
(e.g. CORBA collocation \cite{ObjectInterconnections18, wang00optimizing}), but
compared to pure local method invocations, a noticeable overhead still remains. 


%==============================================================================
\section{Local components}
%==============================================================================

To implement thin components in the same address space we need a specific
component model that defines local components and their interconnections. In
Java already exists such a thin component model (JavaBeans
\cite{Englander1997}), but in other languages like C++ and Python there are no
such specifications.

There is a need for a language independent local component model with
mappings to many different programming languages. Instead of inventing another
new component model, we use the CORBA Component Model in a local manner.


%==============================================================================
\section{Local component adapters}
%==============================================================================

The CCM specification hides most of the CORBA programming details from the
application developer, but it still forces developers to deal with CORBA
references, data types and memory management. While the OMG Java mapping
\cite{OMGIDL2Java} is acceptable in this regard, the C++ mapping
\cite{OMGIDL2Cpp} does not use advantages of standard C++ like strings,
vectors and lists.

To improve the usability of CCM, we proposed a new approach
that provides an easy way to implement components without having to pay 
attention to CORBA details.
We implement all business logic in local components without CORBA.
As shown in Fig.~\ref{LcacOverview}, these local components can be embedded
in a regular CORBA component using a set of adapters $A$.

\begin{figure}[!htb]
    \begin{center}
        \includegraphics [width=7cm,angle=0] {figures/LCAC_Overview}
        \caption{Local component adapter}
        \label{LcacOverview}
    \end{center}
\end{figure}

For every IDL interface or component, we generate corresponding interfaces in the
native implementation language. Adapter classes \cite{Gamma95} provide the
CORBA mapping, and link the business logic to a corresponding CORBA component.

As shown in Fig.~\ref{LcacLayerModel}, there is a local path and a remote path
between two component implementations. When running components in the same
address space, there is no need for CORBA communication. Using the local path
for connecting components significantly reduces the communication overhead.

\begin{figure}[!htb]
    \begin{center}
        \includegraphics [width=8cm,angle=0] {figures/Adapter1}
        \caption{Local vs. remote connection}
        \label{LcacLayerModel}
    \end{center}
\end{figure}

It's important to keep in mind that these adapters do not change the client's
view of a component, so our approach still conforms to the CCM standard.



%==============================================================================
\section{Assemblies of local and remote components}
%==============================================================================

An important issues in {\it Component--Based Software Engineering}
(CBSE) \cite{Szyperski02,IVICA2002,CBSE2001,HerzumSims99} is component granularity. 
Fat components increase runtime performance, but their reuse is limited. 
On the other hand, thin components lead to significant communication overhead 
but are easy to reuse.

Enterprise JavaBeans (EJB), for example, also have a local component concept 
\cite{EJBSpecificationV2_1}: 
A whole bean can be declared as local or
remote by extending the {\tt EJBHome, EJBObject} interfaces or the 
{\tt EJBLocalHome, EJBLocalObject} interfaces respectively.

As with local EJBs, we use local components for thin components located in
the same address space to improve performance and reusability. In contrast to
local EJBs, however, we do not have to decide between a local or remote
component because we always implement a local one. After writing the business
logic we can leave some ports local while some other ports are made remotely
accessible by adding a remote adapter. 

Note that the decision between using the
local or remote adapters does not affect the implementation of the business
logic; in other words we can scale the remote accessibility of a component port
by port at deployment time.
With this approach, we can use the same component model for a wide range of
component implementations and programming languages.


A component assembly can be described as a directed graph where the nodes are component
instances and the edges are receptacle to facet connections.
We can build assemblies of local and remote components using the {\it Session Facade} 
pattern \cite{Marinescu02}.

\begin{figure}[!htb]
    \begin{center}
        \includegraphics [width=8cm,angle=0] {figures/AssemblyGraph}
         \caption{Assembly instance graphs}
        \label{instanceGraph}
    \end{center}
\end{figure}

If an instance of a remote session facade component is created, all connected local 
components also have to be instantiated and connected (Fig.~\ref{instanceGraph}). 

In other words, a {\tt create()} call to a remote session facade component's home 
creates an assembly instance where related components are instantiated and connected 
as described in the assembly descriptor. 
A CORBA reference to the assembly instance is returned by the {\tt create()} 
call to the client.

From the client's point of view, a session facade component is a fat remote
component. In fact, this fat component is an assembly instance graph consisting of
a remote session facade and local component instances that ensure easy reuse of 
business code.

