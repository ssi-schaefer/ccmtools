% $Id$

%==============================================================================
\section{CCM Tools}
%==============================================================================


%==============================================================================
\subsection{Extensions to the CORBA Component Model}
%==============================================================================

\dots


%==============================================================================
\subsection{Restrictions to the CORBA Component Model}
%==============================================================================

\dots


\newpage
%==============================================================================
\subsection{The ccmtools.sf.net Project}
%==============================================================================

The CCM Tools are an open source project that consists of 
a set of Java programs and libraries to support component 
development.

\begin{figure}[htbp]
    \begin{center}
        \includegraphics [width=14cm,angle=0] {figures/CcmToolsOverview}
        \caption{CCM Tools framework}
        \label{ccmtools}
    \end{center}
\end{figure}

\noindent
As shown in Fig.~\ref{ccmtools}, the CCM Tools form a component development 
framework that is flexible and extensible.
Using a well defined {\bf CCM Model}, we can separate component descriptions 
from code generator tools. Therefore, we can add new description methods (e.g.
UML profiles) or code generator tools (e.g. a Java Component Generator) without 
changing other tools.
The current framework contains the following parts:

\begin{description}
\item [uml2idl generator.]
Using the {\it UML profile for CCM} \cite{UML-CCM-Profile} we can model 
component definitions with UML tools. From a given UML XMI file, the uml2idl 
generator extracts an IDL3 and an OCL file. 
While the IDL3 file is used to generate component logic, the OCL file is input
for a {\it Design by Contract} (DbC) generator.

\item [CCM metamodel library.]
The CCM specification defines an {\it Interface Repository Metamodel} 
corresponding to the IDL3 syntax. We implemented a CCM Metamodel Library that 
allows creation and manipulation of CCM models. Using this library we can 
clearly separate parser tools and code generators. The parser creates a 
model object for every part of the source code matching an IDL grammar rule.

\item [idl3 parser.]
The idl3 Parser reads IDL3 files, checks their syntax 
and creates a CCM model in memory using the CCM Metamodel Library. This CCM
model is the starting point for all code generator tools in the framework.

\item [idl3 generator.]
To test the functionality of the CCM Metamodel Library and the IDL3 Parser we 
implemented an IDL3 Generator that writes out a CCM model into corresponding 
IDL3 Files.

\item [idl3mirror generator.]
We use a {\it Test-Driven Development} (TDD) strategy to develop and to specify
components. Our component unit test uses a
mirror component to connect all ports of an existing component. The idl3mirror
Generator creates the IDL3 syntax definition of this mirror component.


\item [OCL metamodel library.]
We have implemented an OCL metamodel library that is used by the OCL parser
to create an OCL model in memory.
This in memory OCL model can be used by generator tools to retrieve
information for code generation.

\item [OCL parser.]
From an OCL source file, an OCL parser performs some syntax checks and
establishes an OCL model in memory using the OCL metamodel library. 
This model is the starting point for code generator tools based on OCL. 

\item [c++local generator.]
The component model is realized by the component logic that implements the
operations for providing, using and connecting components by their facets and
receptacles. We implemented a generator tool that creates the local component
logic starting from a given CCM model. After generating the component logic, 
component developers can write business logic into the generated
component skeletons.

\item [c++dbc generator.]
From an OCL model and a CCM model this generator creates a set of
Design by Contract adapter for local C++ components.
These adapters can be used alternatively to the regular local component
adapters to support DbC.

\item [c++local-test generator]
After building a local C++ component we can run a unit test procedure. To
provide a suitable test environment we generate a test client implementation
that instantiates a component as well as its mirror component and manages this 
component unit test. 

\item [idl2 generator.]
To implement CORBA components, the IDL3 source code is mapped to IDL2 
that can be
processed by some {\bf IDL2 Compiler} (currently we use Mico ORB). The
transformation from IDL3 to IDL2 extends the IDL2 source code with 
operations needed for navigation between components and their ports (equivalent 
operations). 
We support this transformation with an idl2 generator that creates IDL2 files 
from an existing CCM model.

\item [c++remote component generator.]
Local components only work in a common address space and must be
implemented in the same programming language. To overcome these limitations, we
can generate remote component logic that interfaces the local components with 
CORBA.
The remote component logic is thus a superset of the local component logic. Note
that the external view of such a remote component conforms to the LwCCM
specification.

\item [c++remote-test generator.]
After generation of remote components, we can also run a remote test procedure.
In parallel to the local test environment, we can use the mirror component test 
concept in the remote case too.
Thus, a generated remote test client connects a remote component with its
mirror component and manages this remote unit test. 
\end{description}


%==============================================================================
\subsection{CCM Tools runtime architecture}
%==============================================================================

The CCM Tools source code is available from {\tt ccmtools.sourceforge.net}
and is organized in the following structure (Fig.~\ref{ccmtools-structure}):

\begin{figure}[htbp]
    \begin{center}
        \includegraphics [width=10cm,angle=0] {figures/CcmToolsArchitecture}
        \caption{CCM Tools structure}
        \label{ccmtools-structure}
    \end{center}
\end{figure}

\begin{description}
\item [CCM Tools:]
This module contains all the tools and libraries that build the CCM Tools 
framework in form of Java source code.

\item [C++ Runtime Environment:]
This module contains C++ source code that builds the runtime environment for
generated components (e.g. C++ smart pointer library, CCM header files, 
HomeFinder implementations etc.).
Remote components are also based on {\bf Mico} \cite{MicoORB} an open source 
Object Request Broker (ORB) implementation.

\item [Java Runtime Environment:]
This module contains code that is used by Java clients
as a runtime library to access remote CCM components.

\end{description}

\noindent
Remote components are based on distributed computing middleware. We 
use {\bf MICO} an open source {\it Object Request Broker} (ORB) implementation 
\cite{mico} as part of the component's runtime environment.