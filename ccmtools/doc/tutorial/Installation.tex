% $Id$
%==============================================================================
\section{CCM Tools installation}
%==============================================================================

%------------------------------------------------------------------------------
\subsection{Prerequisites}
%------------------------------------------------------------------------------

To install the CCM Tools, the following programs must be available:
\begin{description}
\item [Java SDK $\ge$ 1.4] ({\tt http://java.sun.com/j2se})
\item [Apache Ant $\ge$ 1.5.3] ({\tt http://ant.apache.org})
\item [Python $\ge$ 2.2] ({\tt http://python.org})
\item [cpp $\ge$ 2.96] ({\tt http://www.gnu.org})
\end{description}

To build the generated local C++ components, we also need:
\begin{description}
\item [Confix $\ge$ 1.3pre14] ({\tt http://confix.sourceforge.net})
\item [gcc $\ge$ 2.95.3] ({\tt http://www.gnu.org})
\end{description}

To generate and build remote C++ components, we additionally need:
\begin{description}
\item [mico $\ge$ 2.3.10] ({\tt http://www.mico.org/})
\end{description}


%------------------------------------------------------------------------------
\subsection{How to get it}
%------------------------------------------------------------------------------

The project is hosted at Sourceforge ({\tt http://ccmtools.sf.net}). See the web
site for releases and announcements.

You can also subscribe to the {\tt ccmtools-announce} mailing list for CCM Tools
release announcements. The {\tt ccmtools-users} mailing list provides a forum
for discussion about using the CCM Tools.



%------------------------------------------------------------------------------
\subsection{Source distribution}
%------------------------------------------------------------------------------
Installing of the CCM Tools framework requires the following steps once you get 
the source code. 
(Pretend that the source tarballs you got were {\tt ccmtools-A.B.X.tar.gz},
{\tt cpp-environment-A.B.X.tar.gz} and {\tt java-environment-A.B.X.tar.gz}).

\begin{small}
\begin{verbatim}
 ~> tar xvzf ccmtools-A.B.X.tar.gz
 ~> tar xvzf cpp-environment-A.B.X.tar.gz
 ~> tar xvzf java-environment-A.B.X.tar.gz
\end{verbatim}
\end{small}

Now you have the following directory structure: 
\begin{small}
\begin{verbatim}
  |-- ccmtools/
  |-- cpp-environment/
  `-- java-environment/

\end{verbatim}
\end{small}


%------------------------------------------------------------------------------
\subsubsection{Install java-environment package:}
%------------------------------------------------------------------------------
To build and install the java-environment we use Ant. 
\begin{small}
\begin{verbatim}
 ~> cd java-environment
 ~> ant install 
 ~> cd ..
\end{verbatim}
\end{small}
The java-environment is ready to be used by the ccmtools package!

%------------------------------------------------------------------------------
\subsubsection{Install cpp-environment package:}
%------------------------------------------------------------------------------
We use Confix as C++ build tool. Thus, we have to create a CCM Tools profile
in the {\tt .confix} file.
\begin{small}
\begin{verbatim}
ccm_tools_profile = {
    'PREFIX': '<CCM_INSTALL_PATH>',        # use your own path!
    'BUILDROOT': '/tmp',                   # use your own path!
    'ADVANCED': 'true',
    'CONFIX': {
    },
    'CONFIGURE': {
       'ENV': {
          'CC': '/usr/local/gcc/bin/gcc',  # use your own path!
          'CXX': '/usr/local/gcc/bin/g++', # use your own path!    
          'CFLAGS': "-g -O0 -Wall",
          'CXXFLAGS': "-g -O0 -Wall",
          },
       # use your own mico install path!
       'ARGS': ['--with-external_mico=/usr/local/mico']
       },
    }

PROFILES = {
    'ccmtools': ccm_tools_profile
}
\end{verbatim}
\end{small}
It's important that you substitute your own paths in the {\tt .confix} file.
You can also set the {\tt ccm\_tools\_profile} as default profile.

Now we are ready to install the cpp-environment needed to use local C++ components.
\begin{small}
\begin{verbatim}
 ~> cd cpp-environment
 ~> confix.py --packageroot=`pwd`/WX_Utils --bootstrap --configure  \
              --make --targets="install"
 ~> confix.py --packageroot=`pwd`/CCM_Local --bootstrap --configure \
              --make --targets="install"
\end{verbatim}
\end{small}

Finally, we install the cpp-environment needed to use remote C++
components.

\begin{small}
\begin{verbatim}
 ~> confix.py --packageroot=`pwd`/external --bootstrap --configure  \
              --make --targets="install" 
 ~> confix.py --packageroot=`pwd`/CCM_Remote --bootstrap --configure 
              --make --targets="install"
 ~> cd ..
\end{verbatim}
\end{small}
 
The cpp-environment is ready to be used from generated C++ components!


%------------------------------------------------------------------------------
\subsubsection{Install ccmtools package:}
%------------------------------------------------------------------------------
The ccmtools package can be installed with the following procedure:
\begin{small}
\begin{verbatim}
 ~> cd ccmtools
 ~> autogen.py --prefix=<CCM_INSTALL_PATH> 
 ~> make; make install   
 ~> cd ..
\end{verbatim}
\end{small}
 
Finally, we set some environment variables (these settings are usually added to 
{\tt .bashrc}):
\begin{small}
\begin{verbatim}
 # For using local components: 
 ~> export CCMTOOLS_HOME=<CCM_INSTALL_PATH>
 ~> export PATH=$CCMTOOLS_HOME/bin:$PATH
 ~> export CLASSPATH=$CCMTOOLS_HOME/share/java/ccmtools.jar:\
                     $CCMTOOLS_HOME/share/java/antlr.jar:\
                     $CCMTOOLS_HOME/share/java/mdr01.jar:\
                     $CCMTOOLS_HOME/share/java/uml2idl.jar:\
                     $CCMTOOLS_HOME/share/java/oclmetamodel.jar:\
                     $CLASSPATH

# For using remote components too:
 ~> export CCM_COMPONENT_REPOSITORY=${CCMTOOLS_HOME}
 ~> export CCM_NAME_SERVICE=corbaloc:iiop:1.2@localhost:5050/NameService
\end{verbatim}
\end{small}

OK, that's it! \\
Now we are ready to create and run components using the CCM Tools framework.
