% $Id$
%==============================================================================
\section{CCM Tools installation}
%==============================================================================

%------------------------------------------------------------------------------
\subsection{Prerequisites}
%------------------------------------------------------------------------------

To install the CCM Tools, the following programs must be available:
\begin{description}
\item [Java SDK $\ge$ 1.4] ({\tt http://java.sun.com/j2se})
\item [Apache Ant $\ge$ 1.5.3] ({\tt http://ant.apache.org})
\item [Python $\ge$ 2.2] ({\tt http://python.org})
\item [cpp $\ge$ 2.96] ({\tt http://www.gnu.org})
\end{description}

To build the generated local C++ components, we also need:
\begin{description}
\item [Confix $\ge$ 1.3pre14] ({\tt http://confix.sourceforge.net})
\item [gcc $\ge$ 2.95.3] ({\tt http://www.gnu.org})
\end{description}

To generate and build remote C++ components, we additionally need:
\begin{description}
\item [mico $\ge$ 2.3.10] ({\tt http://www.mico.org/})
\end{description}


%------------------------------------------------------------------------------
\subsection{How to get it}
%------------------------------------------------------------------------------

The project is hosted at Sourceforge ({\tt http://ccmtools.sf.net}). See the web
site for releases and announcements.

You can also subscribe to the {\tt ccmtools-announce} mailing list for CCM Tools
release announcements. The {\tt ccmtools-users} mailing list provides a forum
for discussion about using the CCM Tools.



%------------------------------------------------------------------------------
\subsection{Source distribution}
%------------------------------------------------------------------------------
Installing of the CCM Tools framework requires the following steps once you get 
the source code. 
(Pretend that the source tarballs you got were {\tt ccmtools-A.B.X.tar.gz},
{\tt cpp-environment-A.B.X.tar.gz} and {\tt java-environment-A.B.X.tar.gz}).

\begin{small}
\begin{verbatim}
 $ tar xvzf ccmtools-A.B.X.tar.gz
 $ tar xvzf cpp-environment-A.B.X.tar.gz
 $ tar xvzf java-environment-A.B.X.tar.gz
\end{verbatim}
\end{small}

Alternatively, you can check out an up-to-date version from CVS:
\begin{small}
\begin{verbatim}
 $ cvs -d :pserver:anonymous@cvs.sf.net:/cvsroot/ccmtools login
 Password: <press enter>
 $ cvs -d :pserver:anonymous@cvs.sf.net:/cvsroot/ccmtools co ccmtools
 $ cvs -d :pserver:anonymous@cvs.sf.net:/cvsroot/ccmtools co cpp-environment
 $ cvs -d :pserver:anonymous@cvs.sf.net:/cvsroot/ccmtools co java-environment
\end{verbatim}
\end{small}

Now we should have the following directory structure: 
\begin{small}
\begin{verbatim}
  |-- ccmtools/
  |-- cpp-environment/
  `-- java-environment/
\end{verbatim}
\end{small}


%------------------------------------------------------------------------------
\subsubsection{Install java-environment package:}
%------------------------------------------------------------------------------
To build and install the java-environment we use Ant:
\begin{small}
\begin{verbatim}
 $ cd java-environment
 $ ant install -Dprefix=<install-path>
 
 # Don't forget to set the CLASSPATH environment veriable
 $ export CLASSPATH=<install-path>/lib/antlr.java:   \
                    <install-path>/lib/dtd2java.jar: \
                    <install-path>/lib/jdom.jar:     \     
                    <install-path>/lib/mdr01.jar:    \
                    <install-path>/lib/uml2idl.jar:  \
                    $CLASSPATH
\end{verbatim}
\end{small}



%------------------------------------------------------------------------------
\subsubsection{Install ccmtools package:}
%------------------------------------------------------------------------------
To build and install the  ccmtools package we also use Ant:
\begin{small}
\begin{verbatim}
 # For using local components: 
 $ cd ccmtools
 $ ant install -Dprefix=<CCM_INSTALL_PATH>

 # Don't forget to set the CLASSPATH environment veriable
 $ export CCMTOOLS_HOME=<CCM_INSTALL_PATH>
 $ export CLASSPATH=<CCM_INSTALL_PATH>/lib/ccmtools.jar:    \
                    <CCM_INSTALL_PATH>/lib/oclmetamodel.jar
 $ export PATH=$CCMTOOLS_HOME/bin:$PATH     

 # For using remote components too: 
 $ export CCM_COMPONENT_REPOSITORY=${CCMTOOLS_HOME} 
 $ export CCM_NAME_SERVICE=corbaloc:iiop:1.2@localhost:5050/NameService
\end{verbatim}
\end{small}
 
OK, that's it! \\
Now we are ready to generate CCM components using the CCM Tools framework.


%------------------------------------------------------------------------------
\subsubsection{Install cpp-environment package:}
%------------------------------------------------------------------------------
As shown in Fig.~\ref{ccmtools-structure}, to compile and run generated CCM
components, we need a C++ runtime library called cpp-environment.
To build this cpp-environment package as well as all components we will generate,
Confix is used as C++ build tool. 

First, we have to create a CCM Tools profile in Confix' configuration file
({\tt .confix}), as described in the Confix manual ;-)
\begin{small}
\begin{verbatim}
ccm_tools_profile = {
    'PREFIX': '<CCM_INSTALL_PATH>',        # use your own path!
    'BUILDROOT': '/tmp',                   # use your own path!
    'ADVANCED': 'true',
    'CONFIX': {
    },
    'CONFIGURE': {
       'ENV': {
          'CC': '/usr/local/gcc/bin/gcc',  # use your own path!
          'CXX': '/usr/local/gcc/bin/g++', # use your own path!    
          'CFLAGS': "-g -O0 -Wall",
          'CXXFLAGS': "-g -O0 -Wall",
          },
       # use your own mico install path!
       'ARGS': ['--with-external_mico=/usr/local/mico']
       },
    }

PROFILES = {
    'ccmtools': ccm_tools_profile,
    'default' : ccm_tools_profile
}
\end{verbatim}
\end{small}
It's important that you substitute your own paths in the {\tt .confix} file. \\
Note that you can set the {\tt ccm\_tools\_profile} as default profile.

At last, we are ready to install the cpp-environment:
\begin{small}
\begin{verbatim}
 $ cd cpp-environment

 # Install the local component's environment:
 $ confix.py --packageroot=`pwd`/WX_Utils --bootstrap --configure \
             --make --targets="all install"
 $ confix.py --packageroot=`pwd`/CCM_Local --bootstrap --configure \
             --make --targets="all install"

 # Install the remote component's environment:
 $ confix.py --packageroot=`pwd`/external --bootstrap --configure \
             --make --targets="all install"
 $ confix.py --packageroot=`pwd`/CCM_Remote --bootstrap --configure \
             --make --targets="all install"
\end{verbatim}
\end{small}

Perfect, all tools and libraries have been installed and are ready to work!





