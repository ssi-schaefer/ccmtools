%==============================================================================
\chapter{Remote and Local Components}
%==============================================================================
\begin{flushright}
{\it }
\end{flushright}

%==============================================================================
\section{Introduction}
%==============================================================================

Up to now, we used only local components and assemblies. To develop distributed
applications we need a way to communicate between process boundaries. Remember,
we are using the CORBA component model, so why do we not use CORBA for
communication?

As described in the appendix, we use the {\it Local Component Adapter Concept}
(LCAC) to adapt the local C++ components to the remote CORBA components. The
remote component is a wrapper that uses the local component. Note that the local
component will not be changed in any way, the local or remote communication is
transparent to the business logic hosted in the local component.


%==============================================================================
\section{The first remote component}
%==============================================================================

These tools are still under construction!



%==============================================================================
\section{Assemblies of remote and local components}
%==============================================================================

These tools are still under construction!


