% $Id$
%==============================================================================
\chapter{Connecting Components by Facet and Receptacle}
%==============================================================================
\begin{flushright}
{\it }
\end{flushright}


%------------------------------------------------------------------------------
\section{Overview}
%------------------------------------------------------------------------------
The CORBA component model also defines the concept of receptacles that are
connection points for facet references.
This section describes an example that extends the previous example with
a {\tt Display} component which is connected to the {\tt Hello} component,
as shown in Fig.~\ref{ConnectComponents}.

\begin{figure}[htbp]
    \begin{center}
        \includegraphics [width=10cm,angle=0] {ConnectComponents}
        \caption{Connecting components}
        \label{ConnectComponents}
    \end{center}
\end{figure}

The test client uses the component's home to create component instances.
The component instances are used for configuration and to provide facets.
To connect the two component instances, the {\tt LCD} facet is connected to
the {\tt LCD} receptacle.
Finally, the test client calls operations on the {\tt Console} facet that
are delegated to the {\tt Display} component.


%------------------------------------------------------------------------------
\section{Definition of two components}
%------------------------------------------------------------------------------
We define a new component called {\tt Display} that provides an {\tt LCD}
interface as a facet.
The {\tt Hello} component uses the {\tt LCD} interface as an receptacle.
Note that a facet to receptacle connection can only be established when
both ports handle the same interface.

\begin{verbatim}
    interface LCD {
      void display_text(in string s);
    };

    component Display {
      attribute string prompt;
      provides LCD lcd;
    };

    home DisplayHome manages Display {
    };


    interface Console {
      long println(in string s2);
    };

    component Hello {
      provides Console console;
      uses LCD lcd;
    };

    home HelloHome manages Hello {
    };
\end{verbatim}

\noindent
As long as we define the components in the same file, we have the same file structure
as before:  
\begin{verbatim}
        hello/
        |-- hello.idl
\end{verbatim}



%------------------------------------------------------------------------------
\section{Implementation of two components}
%------------------------------------------------------------------------------

We start with generating and building the empty components:
\begin{verbatim}
~/hello> ccmtools-c++-generate -c 1.0 -d -p hello1.0 hello.idl
~/hello> ccmtools-c++-make -p hello1.0
\end{verbatim}

\noindent
For each of the components a mirror component has been generated as well as a
test client that connects every component with its mirror component and calls 
the component's operations - Test-Driven Development.

To make it short, we implemented the following methods in the business logic
of the {\tt Display} and the {\tt Hello} component:
\begin{verbatim}
// Display_app.cc
void
lcd_impl::display_text ( const std::string& s )
{
  DEBUGNL ( " lcd_impl->display_text ( s )" );
  cout << component->prompt() << s;
}

// Hello_app.cc
long
console_impl::println ( const std::string& s2 )
{
  DEBUGNL ( " console_impl->println ( s2 )" );
  component->ctx->get_connection_lcd().ptr()->display_text(s2);
  cout << endl;
  return s2.length();	
}
\end{verbatim}

\noindent
We build the components again and install them in the component repository:
\begin{verbatim}
~/hello> ccmtools-c++-make -p hello1.0
~/hello> ccmtools-c++-install -p hello1.0
\end{verbatim}



%------------------------------------------------------------------------------
\section{Write a local test client for two components}
%------------------------------------------------------------------------------

To show the use of both components in an application, we write a test client in the
following file structure:
\begin{verbatim}
    hello/
    |-- hello1.0/
    |-- test/
    |   |-- client/
    |   |   |-- client.cc 
\end{verbatim}

\noindent
The client starts with a lot of include and using namespace statements.
Note that the included header files reflect the used components.
\begin{small}
\begin{verbatim}
#include <localComponents/CCM.h>
#include <CCM_Local/HomeFinder.h>
#include <CCM_Utils/Debug.h>

#include <CCM_Local/CCM_Session_Display/Display_gen.h>
#include <CCM_Local/CCM_Session_Hello/Hello_gen.h>
#include <CCM_Local/CCM_Session_Display/DisplayHome_gen.h>
#include <CCM_Local/CCM_Session_Hello/HelloHome_gen.h>

using namespace std;
using namespace CCM_Utils;
using namespace CCM_Local;
using namespace CCM_Session_Display;
using namespace CCM_Session_Hello;
\end{verbatim}
\end{small}

\noindent
In the main function, after setting the debugging mode, 
both component homes are registered at the home finder.
\begin{small}
\begin{verbatim}
int main ( int argc, char *argv[] )
{
  // Set debugging mode
  Debug::set_global(true);

  // Get in instance of the local HomeFinder and register component homes
  localComponents::HomeFinder* homeFinder = HomeFinder::Instance (  );
  try {                       
    homeFinder->register_home( create_DisplayHomeAdapter(), "DisplayHome" );
    homeFinder->register_home( create_HelloHomeAdapter(), "HelloHome" );
  } catch ( ... )  {
    cout << "Aut'sch: when register homes!" << endl;
    return -1;
  }
\end{verbatim}
\end{small}

\noindent
Now, we can use the home finder to find the homes by name.
We create an instance of each component, get references the the component's
facets and connect the facet of {\tt Display} with the receptacle of {\tt Hello}.
Note that both ports are defined by the same {\tt LCD} interface.
\begin{small}
\begin{verbatim}
  try {
    // Find component home
    SmartPtr<DisplayHome> myDisplayHome (dynamic_cast<DisplayHome*>
      ( homeFinder->find_home_by_name ( "DisplayHome").ptr ()));
    SmartPtr<HelloHome> myHelloHome (dynamic_cast<HelloHome*>
      ( homeFinder->find_home_by_name ( "HelloHome").ptr ()));

    // Create component instance
    SmartPtr<Hello> myHello = myHelloHome.ptr()->create();
    SmartPtr<Display> myDisplay = myDisplayHome.ptr()->create();
 
    // Get facet references and connect facets to receptacles
    SmartPtr<CCM_Console> console = myHello.ptr()->provide_console();	
    SmartPtr<CCM_LCD> lcd = myDisplay.ptr()->provide_lcd();
    myHello.ptr()->connect_lcd(lcd);

    // Configure the component's attribute
    myDisplay.ptr()->prompt("-=> ");

    // Component configuration finished	
    myHello.ptr()->configuration_complete();
    myDisplay.ptr()->configuration_complete();
\end{verbatim}
\end{small}

\noindent
The test client has build an component assembly of two components in memory and
can use its functionality. 
\begin{small}
\begin{verbatim}  
    // Call operations on the component assembly
    cout << "Display Version = " 
         << myDisplay.ptr()->getComponentVersion()  << endl;
    cout << "Hello Version = "  
         << myHello.ptr()->getComponentVersion() << endl;

    console.ptr()->println("Hello from the client");
\end{verbatim}
\end{small}

\noindent
To tear the component assembly town, we disconnect the ports and 
remove the component instances from memory.
We also unregister the component's homes from the home finder.
\begin{small}	
\begin{verbatim}    
    // Disconnect component ports
    myHello.ptr()->disconnect_lcd();    

    // Destroy component instances
    myDisplay.ptr()->remove();
    myHello.ptr()->remove();

    // Unregister component homes
    homeFinder->unregister_home("DisplayHome");
    homeFinder->unregister_home("HelloHome");
  }
  catch ( localComponents::HomeNotFound ) {
    cout << "Aut'sch: can't find a home!" << endl;
    return -1;
  }
  catch ( ... )  {
    cout << "Aut'sch: there is something wrong!" << endl;
    return -1;
  }
  return 0;
}
\end{verbatim}
\end{small}

\noindent
We build the test client with {\tt Confix} and run the binary:
\begin{verbatim}
~/hello/test> confix.py --bootstrap --configure  \
                        --make --targets=install --profile=ccmtools
~/hello/test> test_client_client
\end{verbatim}

\noindent
Isn't it cool?\\
We have run a test client that uses two components which are connected via facet 
and receptacle to build a component assembly.
In the same way, we can create more complex assemblies that form a real
component based application as well.

Look at the printed debugging information to trace the execution thread of the
test client. We can also switch off the messages by setting the debugging mode
in the main function to {\tt false}.






