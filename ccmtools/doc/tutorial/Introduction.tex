% $Id$

%==============================================================================
\chapter{CCM Tools Overview}
%==============================================================================

The CCM Tools are a set of Java programs and libraries as well as Python scripts
that supports component development, based on the {\it CORBA Component
Model} (CCM) \cite{CCMSpecification} and our local component extensions. 

\begin{figure}[htbp]
    \begin{center}
        \includegraphics [width=12cm,angle=0] {ComponentGeneratorTools}
        \caption{CCM Tools framework}
        \label{ccmtools}
    \end{center}
\end{figure}

As shown in Fig.~\ref{ccmtools}, the CCM Tools form a component development framework 
that is flexible and extensible.


\newpage
Using a well defined {\bf CCM model}, we can separate component descriptions from
code generator tools. Therefore, we can add new description methods (e.g.
UML profiles) or code generator tools (e.g. a Java Component Generator) without changing
other tools.

The current CCM Tools framework contains the following tools:
\begin{description}
\item [uml2idl Generator]
Using a UML profile for CCM \cite{} we can model component definitions with
UML tools. From a given UML XMI file a uml2idl generator extracts a IDL3 
and an OCL file. 
While the IDL3 file is used to generate component logic, the OCL file is input
for a Design by Contract (DbC) generator (as described later).

\item [CCM Metamodel Library]
The CCM specification defines an {\it Interface Repository Metamodel} corresponding
to the IDL3 syntax. We implemented a CCM Metamodel Library that allows creation and
manipulation of CCM models. Using this library we can clearly separate the parser
tools and code generators. The parser creates a model object for every part of the
source code matching an IDL grammar rule.

\item [idl3 Parser]
The idl3 Parser reads IDL3 files, checks the syntax of these source code
and creates a CCM model using the CCM Metamodel Library in the memory. This CCM
model is the starting point for all code generator tools in the framework.

\item [idl3 Generator]
To test the functionality of the CCM Metamodel Library and the IDL3 Parser we implemented
an IDL3 Generator that writes out a CCM model into corresponding IDL3 Files.

\item [idl3mirror Generator]
We use a {\it Test-Driven Development} (TDD) strategy to develop and to specify
components (as described later in this tutorial). Our component unit test uses a
mirror component to connect all ports of an existing component. The idl3mirror
Generator creates the IDL3 syntax definition of this mirror component.


\item [OCL Metamodel Library]
We have implemented an OCL metamodel library that is used by the OCL parser
to create an OCL model in memory.
This in memory OCL model can be used by generator tools to retrieve
information for code generation.

\item [OCL Parser]
From an OCL source file, an OCL parser performs some syntax checks and
establishes an OCL model in memory using the OCL metamodel library. 
This model is the starting point for code generator tools based on OCL. 

\item [c++local Generator]
The component model is realized by the component logic that implements the
operations for providing, using and connecting components by their facets and
receptacles. We implemented a generator tool that creates the local component
logic starting from a given CCM model. After generating the component logic, 
component developers can write business logic into the generated
component skeletons.

\item [c++dbc Generator]
From an OCL model and a CCM model this generator creates a set of
Design by Contract adapter for local C++ components.
These adapters can be used alternatively to the regular local component
adapters to support DbC.

\item [c++local-test Generator]
After building a local C++ component we can run a unit test procedure. To
provide a suitable test environment we generate a test client implementation
that instantiates a component as well as its mirror component and manages this 
component unit test. 

\item [idl2 Generator]
To implement CORBA components the IDL3 source code is mapped to IDL2 that can be
processed by a usual {\bf IDL2 Compiler} (currently we use Mico ORB). The
transformation from IDL3 to IDL2 extends the IDL2 source code with some operations 
needed for navigation between components and their ports (equivalent operations). 
We support this transformation with a idl2 Generator that creates IDL2 files from 
a existing CCM model.

\item [c++remote Component Generator]
The local components only works in a common address space and must be
implemented in the same programming language. To overcome these limitations we
can generate remote component logic that interfaces the local components with CORBA.
The remote component logic is thus a superset of the local component logic. Note
that the external view of such a remote component quite conforms to the CCM
specification.

\item [c++remote-test Generator]
After generation of remote components we can also run a remote test procedure.
In parallel to the local test environment, we can use the mirror component test 
concept in the remote case too.
Thus, a generated remote test client connects a remote component with its
mirror component and manages this remote unit test. 
\end{description}


%==============================================================================
\section{CCM Tools runtime architecture}
%==============================================================================

The CCM Tools source code is available from {\tt ccmtools.sourceforge.net}
and is organized in the following structure (Fig.~\ref{ccmtools-structure}):

\begin{figure}[htbp]
    \begin{center}
        \includegraphics [width=10cm,angle=0] {CcmToolsArchitecture}
        \caption{CCM Tools structure}
        \label{ccmtools-structure}
    \end{center}
\end{figure}

\begin{description}
\item [java-environment:]
This module contains Java code and libraries that are used by the CCM Tools
but can also be used independently (e.g. a UML parser that is part of the 
uml2idl generator, a dtd2java generator that can generate XML parsers with a 
model library to handle XML files in memory, etc.).
  
\item [cpp-environment:]
This module contains C++ source code that builds a runtime environment for
C++ components (e.g. C++ smart pointer library, CCM header files, HomeFinder
implementations etc.).

\item [ccmtools:]
This module contains all the CCM Tools in form of Java source code and Python
scripts.
Some CCM Tools need jar files from java-environment to run, while the generated C++ 
components depend only on cpp-environment libraries. 
Remote components are also based on {\bf Mico} \cite{MicoORB} an open source Object 
Request Broker (ORB) implementation.

\end{description}

\newpage
% $Id$
%==============================================================================
\chapter{Installation}
%==============================================================================

%------------------------------------------------------------------------------
\section{Prerequisites}
%------------------------------------------------------------------------------

To install the CCM Tools, the following programs must be available:
\begin{description}
\item [Java SDK $\ge$ 1.4] ({\tt http://java.sun.com/j2se/})
\item [Python $\ge$ 2.2] ({\tt http://python.org/})
\item [cpp $\ge$ 2.96] ({\tt http://www.gnu.org/})
\end{description}

To build the generated local C++ components, we need:
\begin{description}
\item [Confix $\ge$ 1.3pre14] ({\tt http://confix.sourceforge.net/})
\item [gcc $\ge$ 2.95.3] ({\tt http://www.gnu.org/})
\end{description}

To generate and build a remote C++ component, we need:
\begin{description}
\item [mico $\ge$ 2.3.10] ({\tt http://www.mico.org/})
\end{description}


%------------------------------------------------------------------------------
\section{How to get it}
%------------------------------------------------------------------------------

The project is hosted at Sourceforge ({\tt http://ccmtools.sf.net}). See the web
site for releases and announcements.

You can also subscribe to the {\tt ccmtools-announce} mailing list for CCM Tools
release announcements. The {\tt ccmtools-users} mailing list provides a forum
for discussion about using the CCM Tools.

%------------------------------------------------------------------------------
\section{Source distribution}
%------------------------------------------------------------------------------

%------------------------------------------------------------------------------
\subsection{Install the CCM-Tools}
Installing of the CCM Tools requires the following steps once you get the source
code. (Pretend that the source tarball you got was {\tt ccmtools-A.B.X.tar.gz}.)
You will need to configure, build, and install the binary files, and then set
environment variables appropriately:
\begin{small}
\begin{verbatim}
  ~> tar xvzf ccmtools-A.B.X.tar.gz
  ~> cd ccmtools-A.B.X

  ~> # CCMTOOLS_HOME defines the directory the ccmtools are installed to.
  ~> export CCMTOOLS_HOME=<ccm_tools_path>
  ~> # CCM_COMPONENT_REPOSITORY defines the directory where the generated
  ~> # components will be installed.
  ~> export CCM_COMPONENT_REPOSITORY=$CCMTOOLS_HOME

  ~> ./autogen.py --prefix=$CCMTOOLS_HOME
  ~> make
  ~> make install

  ~> export PATH=<ccm_tools_path>/bin:$PATH
  ~> export CLASSPATH=$CCMTOOLS_HOME/share/java/ccmtools.jar: \
                      $CCMTOOLS_HOME/share/java/antlr.jar:    \
                      $CLASSPATH

  ~> # To test the installed ccmtools and the set paths, type:
  ~> ccmtools-generate --version
\end{verbatim}
\end{small}



You can also use the binary distribution, which includes a minimal set of Jar
archives, templates, and documents. To install this distribution, download the
binary distribution (again assume {\tt ccmtools-A.B.X.bin.tar.gz}) file and do
the following:
\begin{small}
\begin{verbatim}
  ~> tar xvzf ccmtools-A.B.X.bin.tar.gz
  ~> cd ccmtools-A.B.X

  ~> export CCMTOOLS_HOME=`pwd`
  ~> export CCM_COMPONENT_REPOSITORY=$CCMTOOLS_HOME

  ~> export PATH=`pwd`/bin:$PATH
  ~> export CLASSPATH=`pwd`/share/java/ccmtools.jar: \
                      `pwd`/share/java/antlr.jar:    \
                      $CLASSPATH
\end{verbatim}
\end{small}

%------------------------------------------------------------------------------

\subsection{Install the local and remote component environment}

To build and run local and remote components a set of libraries and header files
are needed. These files must be installed only once and are commonly used from all
generated components.

\subsubsection{Install the local component environment}
When using local components only, we install the local component environment with
a single Confix call, and skip the rest of the chapter. 
\begin{small}
\begin{verbatim}
  ~> cd ccmottls-A.B.C/environment
  ~> confix.py --bootstrap --configure --profile="ccmtools" \
               --packageroot="local" --make --targets=install
\end{verbatim}
\end{small}


\subsubsection{Create a new Confix profile}
The CCM-Tools use Confix to build and package generated C++ components. As described
in the Confix manual, there must be a ~/.confix file that contains one or more
Confix profiles. In the context of CCM-Tools, we define a new Confix profile:
\begin{small}
\begin{verbatim}
    # Create a new Confix profile within ~/.confix

    ccm_tools_profile = {
     # Use the same path as defined in CCM_COMPONENT_REPOSITORY for PREFIX 
    'PREFIX': <component_install_path>, 
    'BUILDROOT': '/tmp',
    'ADVANCED': 'true',
    'CONFIX': {},
    'CONFIGURE': {
       'ENV': {
          # Use your own path to gcc and g++
          'CC': '/usr/local/gcc/bin/gcc',
          'CXX': '/usr/local/gcc/bin/g++',
          'CFLAGS': "-g -O0 -Wall -DCCM_DEBUG",
          'CXXFLAGS': "-g -O0 -Wall -DCCM_DEBUG",
          },
       # Use your own mico install path
       'ARGS': ['--with-external_mico=</usr/local/mico>']
       },
    }
    PROFILES = {
    #...
    'ccmtools': ccm_tools_profile
    }
\end{verbatim}
\end{small}


\subsubsection{Install the remote component environment}
To build real CORBA components we need some additional libraries including the 
mico ORB.  
\begin{small}
\begin{verbatim}
  # First we have to tell Confix where to find the mico libraries
  > cd ccmtools-0.3.3/environment/external
  > confix.py --bootstrap --configure --profile="ccmtools" \
              --packageroot="external" --make --targets=install    

  # After that, we can install the remote component environment 
  > confix.py --bootstrap --configure --profile="ccmtools" \
              --packageroot="remote" --make --targets=install

  # Now we can run the mico NameService
  > nsd -ORBIIOPAddr inet:localhost:5050 -ORBIIOPVersion 1.2

  # The remote test client needs the CCM_NAME_SERVICE environment variable to 
  # find the NameService at runtime.
  > export CCM_NAME_SERVICE=corbaloc:iiop:1.2@localhost:5050/NameService

  # Finally, we can check the mico NameService with the nsadmin tool
  > nsadmin -ORBInitRef NameService=corbaloc:iiop:1.2@localhost:5050/NameService
    nsadmin> help
\end{verbatim}
\end{small}

OK, that's it! \\
Now we are ready to create and run components using the CCM-Tools framework.

