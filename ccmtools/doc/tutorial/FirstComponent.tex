% $Id$
%==============================================================================
\chapter{Component Development}
%==============================================================================


%------------------------------------------------------------------------------
\section{Component development process overview}
%------------------------------------------------------------------------------

This section shows how to implement a simple component using the CCM
Tools framework. 
Before you start with component development, make sure that the CCM Tools are
installed in a proper way.

\vspace{3mm}
As shown in Fig.~\ref{fig:development-process}, the component development 
process defines different roles that reflect different abstraction levels of 
components:

\begin{figure}[htbp]
    \begin{center}
        \includegraphics [width=10cm,angle=0] {DevelopmentProcess}
        \caption{Component development process}
        \label{fig:development-process}
    \end{center}
\end{figure}
\begin{description}
\item [Component Designer]
specifies the structure of the component by defining {\it Interface
Description Language} (IDL) files. The interfaces defined in the IDL source
file(s) are only syntax definitions and do not describe the behavior of the
component. These IDL declarations merely specify the methods that a component is
required to implement. In this way, an IDL file is a sort of protocol
definition for a component.

\item [Component Developer]
implements the component's behavior with respect to the defined interfaces.
\end{description}

Using IDL to specify the outside view of components ensures that we can connect
these black boxes that may be implemented in different programming languages.

\newpage
The CCM Tools framework support this development process by providing an IDL 
parser and different code generators. 
Each component has two logical parts 
(Fig.~\ref{fig:componentlogic-businesslogic}), and generated code is used in 
both parts:

\begin{figure}[htbp]
    \begin{center}
        \includegraphics [width=5cm,angle=0] {ComponentLogicBusinessLogic}
        \caption{Component logic with embedded business logic}
        \label{fig:componentlogic-businesslogic}
    \end{center}
\end{figure}

\begin{description}
\item [Component Logic]
is the implementation of the component's structure. It contains mechanisms to
create and destroy component instances as well as access and navigation methods
for finding and manipulating ports and attributes. This code can be completely
generated from IDL descriptions.

\item [Business Logic]
is the implementation of the component's behavior.
Generator tools can create skeleton code, but the component
developer has to implement the business code manually.

Happily, the component developer need not be a CORBA guru to implement these
functions: Generated remote (out--of--process) component logic can handle the
interface with a CORBA server.

Also, because of the nature of a component based application, the component
developer should not implement a given component's business logic with an entire
application in mind. Components are, on the contrary, meant to encapsulate well
defined logical processes without caring about the overall structure of the
application environment.
\end{description}

Finally, the CCM Tools make use of the {\it Confix} build tool to create  
libraries from generated (component logic) and hand-crafted (business logic) 
source code. 
The component libraries are installed in a component repository on the server
side and can be used to assemble modular, component based applications.

\newpage


%------------------------------------------------------------------------------
\section{The designer's job}
%------------------------------------------------------------------------------

Let's start to design the first component. 
In the role of a component designer,
we have to create a bunch of IDL files that describes a component's structure
(Fig~\ref{fig:component-design}). 

\begin{figure}[htbp]
    \begin{center}
        \includegraphics [width=4cm,angle=0] {DesignerTask}
        \caption{Component design}
        \label{fig:component-design}
    \end{center}
\end{figure}

As a simple example, we model a car rental component that can handle customers
and miles. For each customer we can evaluate driven miles and costs incurred. 
Of course, this is not a real world application, but this example is
well suited to demonstrate CCM Tools in action.

We will store all created IDL files in a {\tt example1} directory
\footnote{You can find this example source code in the 
{\tt ccmtools/test/CppGenerator/tutorial/} directory.}:
\begin{small}
\begin{verbatim}
        example1/
        |-- Exceptions.idl
        |-- Customer.idl
        |-- CustomerMaintenance.idl
        |-- CustomerBusiness.idl
        `-- CarRental.idl
\end{verbatim}
\end{small}

Keep in mind that IDL files contain no specification of what exact action will 
be performed.
The component developer bears the responsibility of filling in the functionality
after the component logic has been generated.

\begin{Example}
\begin{minifbox}
\begin{small}
\begin{verbatim}
#ifndef __EXCEPTIONS_IDL__
#define __EXCEPTIONS_IDL__

module BigBusiness {
  exception CreateCustomerException{};
  exception RemoveCustomerException{};
  exception NoCustomerException{};
}; // End of module BigBusiness

#endif
\end{verbatim}
\end{small}
\end{minifbox}
\caption{Exceptions.idl}
\label{example:Exceptions}
\end{Example}
The first code snippet (example \ref{example:Exceptions}) shows the definition 
of three IDL exceptions.
In order to get a namespace, we envelop these definitions using a module called 
{\tt BigBusiness}. 
To prevent IDL files from multiple including, we define {\it include guards} 
within each source file.


\newpage
 
\begin{Example}
\begin{minifbox}
\begin{small}
\begin{verbatim}
#ifndef __CUSTOMER_IDL__
#define __CUSTOMER_IDL__

module BigBusiness {
  struct Customer {
    long id;
    string first_name;
    string last_name;
    double mileage;
  };
  typedef sequence<Customer> CustomerList;
}; // End of module BigBusiness

#endif
\end{verbatim}
\end{small}
\end{minifbox}
\caption{Customer.idl}
\label{example:Customer}
\end{Example}

In {\tt Customer.idl}, we define a structure of basic data types and a sequence 
of this structure type.
Again, there is a {\tt BigBusiness} module and an include guard.

\begin{Example}
\begin{minifbox}
\begin{small}
\begin{verbatim}
#ifndef __CUSTOMER_MAINTENANCE_IDL__
#define __CUSTOMER_MAINTENANCE_IDL__

#include"Exceptions.idl"
#include"Customer.idl"

module BigBusiness {
  interface CustomerMaintenance
  {
    void createCustomer(in Customer person) 
      raises (CreateCustomerException);
    Customer retrieveCustomer(in long id)  
      raises (NoCustomerException);
    CustomerList retrieveAllCustomers()  
      raises (NoCustomerException);
    void updateCustomer(in Customer person)  
      raises (NoCustomerException);
    void deleteCustomer(in long id)  
      raises (RemoveCustomerException);
  };
}; // End of module BigBusiness
#endif
\end{verbatim}
\end{small}
\end{minifbox}
\caption{CustomerMaintenance.idl}
\label{example:}
\end{Example}
Customers must be handled in a database like way. 
Thus, {\tt CustomerMaintenance.idl} contains a 
{\it Create, Retrieve, Update and Delete} (CRUD) interface. 
To pick a particular {\tt Customer}, we use a customer id that is realized by 
an IDL long type.
Methods can throw exceptions in the case of illegal customer ids, create or 
remove problems. 

\newpage

\begin{Example}
\begin{minifbox}
\begin{small}
\begin{verbatim}
#ifndef __CUSTOMER_BUSINESS_IDL__
#define __CUSTOMER_BUSINESS_IDL__

#include "Exceptions.idl"

module BigBusiness {
  interface CustomerBusiness
  {
    attribute double dollars_per_mile;
    void addCustomerMiles(in long id, in double miles) 
      raises(NoCustomerException);
    void resetCustomerMiles(in long id) 
      raises(NoCustomerException);
    double getCustomerMiles(in long id) 
      raises(NoCustomerException);
    double getCustomerDollars(in long id) 
      raises(NoCustomerException);
  };
}; // End of module BigBusiness
#endif
\end{verbatim}
\end{small}
\end{minifbox}
\caption{CustomerBusiness.idl}
\label{example:}
\end{Example}

Beside the pure data manipulation, we also need an interface that defines a
business functionality.
{\tt CustomerBusiness.idl} defines an attribute and some methods that operate
on the given customer data. 
If a parameter represents a wrong customer id, the {\tt NoCustomerException}
will be thrown.

\begin{Example}
\begin{minifbox}
\begin{small}
\begin{verbatim}
#ifndef __CAR_RENTAL_IDL__
#define __CAR_RENTAL_IDL__

#include "CustomerMaintenance.idl"
#include "CustomerBusiness.idl"

module BigBusiness {

  component CarRental 
  { 
    provides CustomerMaintenance maintenance;
    provides CustomerBusiness business;
  };
  
  home CarRentalHome manages CarRental 
  { };

}; // End of module BigBusiness
#endif
\end{verbatim}
\end{small}
\end{minifbox}
\caption{CarRental.idl}
\label{example:component}
\end{Example}

Finally, we collect the included interfaces to a {\tt CarRental} component and 
define a {\tt CarRentalHome} that will be used to create component instances at
runtime.

Well done! We have defined our first component in terms of IDL.


\newpage

%------------------------------------------------------------------------------
\section{The developer's job}
%------------------------------------------------------------------------------

Remember, the component developer implements the behavior of the component with
respect to the defined IDL interfaces (Fig.~\ref{fig:component-development}). 
\begin{figure}[htbp]
    \begin{center}
        \includegraphics [width=8cm,angle=0] {DeveloperTask}
        \caption{Component development}
        \label{fig:component-development}
    \end{center}
\end{figure}

This section shows how the CCM Tools make this job easier.

%------------------------------------------------------------------------------
\subsection{Create component logic from IDL}
%------------------------------------------------------------------------------

In the first step, we bring the IDL3 files in a proper structure that implies 
the separation between interface and component definitions. 
\begin{small}
\begin{verbatim}
~/example1> ccmtools-generate idl3 -o CarRental/idl3 *.idl
\end{verbatim}
\end{small}
As result of this CCM Tools call is the following file structure.
The separation between interface and component related files reflects the
fact that interfaces are used to defined components but are not owned by
these components.
\begin{small}
\begin{verbatim}
CarRental
`-- idl3
    |-- component
    |   `-- BigBusiness
    |       |-- CarRental.idl
    |       `-- CarRentalHome.idl
    `-- interface
        `-- BigBusiness
            |-- CreateCustomerException.idl
            |-- Customer.idl
            |-- CustomerBusiness.idl
            |-- CustomerList.idl
            |-- CustomerMaintenance.idl
            |-- NoCustomerException.idl
            `-- RemoveCustomerException.idl
\end{verbatim}
\end{small}

\newpage
To generate C++ source code from IDL3 we have to call the CCM Tools generator
twice, once for the interfaces and once for the component:
\begin{small}
\begin{verbatim}
~/example1> ccmtools-generate c++local \
                -o CarRental/interface \
                -ICarRental/idl3/interface \
                CarRental/idl3/interface/BigBusiness/*.idl
~/example1> ccmtools-generate c++local -a \
                -o CarRental/component/CarRental \
                -ICarRental/idl3/component -ICarRental/idl3/interface \
                CarRental/idl3/component/BigBusiness/CarRental.idl \
                CarRental/idl3/component/BigBusiness/CarRentalHome.idl
\end{verbatim}
\end{small}

Now, the {\tt CarRental} subdirectory contains the component's source code. 
(To get a description of all {\tt ccmtools-generate} options consult the 
CCM Tools commands section in the tutorial's appendix).
\begin{small}
\begin{verbatim}
CarRental
|-- idl3
|-- component
|   `-- CarRental
|       |-- CCM_Local_BigBusiness_CCM_Session_CarRental
|       |-- CCM_Local_BigBusiness_CCM_Session_CarRental_share
|       `-- impl
|           |-- CarRentalHome_entry.h
|           |-- CarRentalHome_impl.cc
|           |-- CarRentalHome_impl.h
|           |-- CarRental_business_impl.cc
|           |-- CarRental_business_impl.h
|           |-- CarRental_impl.cc
|           |-- CarRental_impl.h
|           |-- CarRental_maintenance_impl.cc
|           |-- CarRental_maintenance_impl.h
|           `-- Makefile.py
`-- interface
    `-- CCM_Local_BigBusiness
\end{verbatim}
\end{small}

The component developer really ought not to care about the generated 
{\tt CCM\_Local\_*} subdirectories (if you don't believe us, though, feel free 
to read through it; just don't edit anything). 
The interesting files are listed in the {\tt impl} directory. 
These files contain skeleton code (that are empty methods) of the component's 
business logic.

\begin{itemize}
\item {\tt CarRentalHome\_entry.h} \\
This file is the business logic's entry point and declares an extern ``C'' 
function that returns a component home instance.
The generated implementation of this function is hosted in the 
{\tt CarRentalHome\_impl.cc} file.

\item {\tt CarRentalHome\_impl.[h,cc]} \\
These files contain the factory methods for component instantiation. 
While user defined factory methods must be implemented manually, the default 
{\tt create} method is already generated.

\item {\tt CarRental\_impl.[h,cc]} \\
These files contain the implementation of supported interfaces as well as the
implementation of component attributes.

\item {\tt CarRental\_business\_impl.[h,cc]} \\
{\tt CarRental\_maintenance\_impl.[h,cc]} \\
These files contain the implementation of defined facets and builds
the component's business logic
(grep for {\tt // TODO : IMPLEMENT ME HERE !}).

\item {\tt Makefile.py} \\
This file is a Confix marker that denotes a directory as 'to compile'.
\end{itemize}

Files in the {\tt impl} directory are protected from overwriting.
Running the generator again results in a bunch of warnings. 
Instead of overwriting, the generator stores the new files with an extension 
({\tt *.new}) into the same directory.
You can make a {\tt diff} to figure out changes. 

\vspace{3mm}
The generated business logic skeletons contains enough code to be compilable.
To build the generated source code, type:
\begin{small}
\begin{verbatim}
# Compile and install component interfaces
$ confix.py --advanced --bootstrap --configure \
            --packageroot=`pwd`/CarRental/interface \
            --make --targets=install

# Compile the components and run the unit test
$ touch CarRental/component/CarRental/Makefile.py
$ confix.py --advanced --bootstrap --configure \
            --packageroot=`pwd`/CarRental/component \
            --make 
\end{verbatim}
\end{small}


%------------------------------------------------------------------------------
\subsection{Test driven development}
%------------------------------------------------------------------------------

Software development is an iterative process. Kent Beck
has proposed a {\it Test Driven Development} (TDD) methodology 
\cite{Beck2003TDD} that starts by
implementing the test~-- before implementing the business logic!

\begin{figure}[!htb]
    \begin{center}
        \includegraphics [width=7cm,angle=0] {TestDrivenDevelopment}
        \caption{Test driven development process}
        \label{fig:test-driven-development}
    \end{center}
\end{figure}

As shown in Fig.~\ref{fig:test-driven-development}, the CCM Tools make use of
the TDD idea in the context of components. For every component $C$, we
generate a mirror component $\overline{C}$. Each input port (receptacle)
of $C$ corresponds to an output port (facet) of $\overline{C}$, and vice versa.

\newpage
To create a mirror component $\overline{C}$ definition from manually written 
IDL3 files, type:
\begin{small}
\begin{verbatim}
~/example1> ccmtools-generate idl3mirror -o CarRental/idl3 *.idl
\end{verbatim}
\end{small}

From the generated IDL3 files you can see that both $\overline{C}$ and $C$ 
uses the same interfaces.
\begin{small}
\begin{verbatim}
CarRental
`-- idl3
    |-- component
    |   `-- BigBusiness
    |       |-- CarRental.idl
    |       |-- CarRentalHome.idl
    |       |-- CarRentalHome_mirror.idl    // new
    |       `-- CarRental_mirror.idl        // new
    `-- interface
        `-- BigBusiness
            |-- CreateCustomerException.idl
            |-- Customer.idl
            |-- CustomerBusiness.idl
            |-- CustomerList.idl
            |-- CustomerMaintenance.idl
            |-- NoCustomerException.idl
            `-- RemoveCustomerException.idl
\end{verbatim}
\end{small}

Now we can use the {\tt c++local} generator again to create C++ component logic 
from the IDL3 mirror files.
\begin{small}
\begin{verbatim}
~/example1> ccmtools-generate c++local -a \
                -o CarRental/component/CarRental_mirror \
                -ICarRental/idl3/component -ICarRental/idl3/interface \
                CarRental/idl3/component/BigBusiness/CarRental_mirror.idl \
                CarRental/idl3/component/BigBusiness/CarRentalHome_mirror.idl
\end{verbatim}
\end{small}
This results in the following file structure:
\begin{small}
\begin{verbatim}
CarRental 
|-- component
|   |-- CarRental
|   `-- CarRental_mirror
|       |-- CCM_Local_BigBusiness_CCM_Session_CarRental_mirror
|       |-- CCM_Local_BigBusiness_CCM_Session_CarRental_mirror_share
|       `-- impl
|           |-- CarRentalHome_mirror_entry.h
|           |-- CarRentalHome_mirror_impl.cc
|           |-- CarRentalHome_mirror_impl.h
|           |-- CarRental_mirror_impl.cc
|           |-- CarRental_mirror_impl.h
|           `-- Makefile.py
\end{verbatim}
\end{small}

The {\tt impl} directory contains generated files of the mirror
component's business logic that can host component test cases.
In the next sections of this tutorial, we will implement test cases as well
as business logic. 

\newpage
Finally, a test client coordinates creating and connecting of component $C$ and 
mirror component $\overline{C}$.
To generate a test client, we type:
\begin{small}
\begin{verbatim}
~/example1> ccmtools-generate c++local-test \
                -o CarRental/component/CarRental_mirror \
                -ICarRental/idl3/component -ICarRental/idl3/interface \
                CarRental/idl3/component/BigBusiness/CarRental.idl
\end{verbatim}
\end{small}

The generated files are stored in the following way:
\begin{small}
\begin{verbatim}
CarRental
|-- component
|   |-- CarRental
|   `-- CarRental_mirror
|       |-- CCM_Local_BigBusiness_CCM_Session_CarRental_mirror
|       |-- CCM_Local_BigBusiness_CCM_Session_CarRental_mirror_share
|       |-- impl
|       `-- test
|           |-- Makefile.py
|           `-- _check_CCM_Local_BigBusiness_CCM_Session_CarRental.cc
\end{verbatim}
\end{small}

The {\tt test} directory contains a generated {\tt \_check\_*} file that 
implements the test client and is also protected from overwriting.



\newpage
%------------------------------------------------------------------------------
\subsection{Write the test first}
%------------------------------------------------------------------------------

Now is a good time to start the development of business code. First, we have to
define the behavior of the business code in an executable semantic. 
In other words, we write a test case in the mirror component 
({\tt CarRental\_mirror\_impl.cc}), as
shown in Example~\ref{example:one-component-mirror-impl}.

\begin{Example}
\begin{minifbox}
\begin{small}
\begin{verbatim}
void CCM_CarRental_mirror_impl::ccm_activate()
  throw ( LocalComponents::CCMException )
{
  try {
    {
      CCM_Local::BigBusiness::Customer person;
      person.id = 1;
      person.first_name = "Franz";
      person.last_name = "Kafka";
      ctx->get_connection_maintenance_mirror()->
           createCustomer(person);
    }
    {
      CCM_Local::BigBusiness::Customer person;
      long id = 1;
      person = ctx->get_connection_maintenance_mirror()->
               retrieveCustomer(id);
      assert(person.id == 1);
      assert(person.first_name == "Franz");
      assert(person.last_name == "Kafka");
    }
  }
  catch(CCM_Local::BigBusiness::CreateCustomerException) {
    cerr << "MAINTENANCE ERROR: Can't create customer!" << endl;
  }
  catch(CCM_Local::BigBusiness::NoCustomerException) {
    cerr << "MAINTENANCE ERROR: no customer found!" << endl;
  }
}
\end{verbatim}
\end{small}
\end{minifbox}
\caption{Test case implementation in the mirror component.}
\label{example:one-component-mirror-impl}
\end{Example}
The test case is implemented in the {\tt ccm\_activate} callback method of the
mirror component because this method is called after creating and connecting $C$
and $\overline{C}$ from the component logic. 

First, we creating a valid {\tt Customer} record and  
use the context object ({\tt ctx}) to get a smart pointer
to $\overline{C}$'s receptacle (which is connected with $C$'s facet) 
before we call the {\tt createCustomer} operation. 
After that, we retrieve the record from the receptacle and compare the stored 
items.

To compile and run the test, we type:
\begin{small}
\begin{verbatim}
~/example1> touch CarRental/component/CarRental_mirror/Makefile.py
~/example1> confix.py --advanced --bootstrap --configure \
                      --packageroot=`pwd`/CarRental/component \
                      --make --targets=check
\end{verbatim}
\end{small}
The test failed! 
That's fine because there is no business code to work with...
%Remember, there is no business code implemented in {\tt createCustomer} and 
%{\tt retrieveCustomer}; they are empty function skeletons. 






%------------------------------------------------------------------------------
\subsection{Write the business logic}
%------------------------------------------------------------------------------

Now we are ready to implement the business logic that satisfies the written 
test case. 
First, we have to implement the customer database - no panic - we 
simple put a C++ {\tt std::vector<>} into the {\tt CarRental\_impl.h} file,
to hold {\tt Customer} entities.
\begin{Example}
\begin{minifbox}
\begin{small}
\begin{verbatim}
class CCM_CarRental_impl : virtual public CCM_CarRental
{
  public:
    std::vector<CCM_Local::BigBusiness::Customer> CustomerDB;
    //...
};
\end{verbatim}
\end{small}
\end{minifbox}
\caption{Business logic header file.}
\label{example:one-component-impl}
\end{Example}

Next, we open {\tt CarRental\_maintenance\_impl.cc} and add some lines, as 
shown in Example~\ref{example:one-component-impl}.
\begin{Example}
\begin{minifbox}
\begin{small}
\begin{verbatim}
void
maintenance_impl::createCustomer ( const Customer& person )
  throw (LocalComponents::CCMException, CreateCustomerException)
{
  component->CustomerDB.push_back(person);
}

Customer
maintenance_impl::retrieveCustomer ( const long id )
  throw (LocalComponents::CCMException, NoCustomerException)
{
  std::vector<CCM_Local::BigBusiness::Customer>::iterator pos;
  for(pos = component->CustomerDB.begin(); 
      pos != component->CustomerDB.end(); ++pos) {
    if(pos->id == id) {
      return *pos;
    }
  }
  throw NoCustomerException();
}
\end{verbatim}
\end{small}
\end{minifbox}
\caption{Business logic implementation.}
\label{example:one-component-impl}
\end{Example}

Instead of a database, we simply use a {\tt std::vector<>} to implement
the {\tt createCustomer} and {\tt retrieveCustomer} methods~-- that's it! 

We'll run the test again to see if it's working:
\begin{small}
\begin{verbatim}
~/example1> confix.py --advanced \
                      --packageroot=`pwd`/CarRental/component \
                      --make --targets=check
\end{verbatim}
\end{small}

Yay, the test passed!



%------------------------------------------------------------------------------
\subsection{Write the test, write the business logic, run the test ...}
%------------------------------------------------------------------------------
The idea of test driven development is to run short cycles of test and
implementation until all required functionality is implemented.
For the {\tt CarRental} component, this section goes through all these 
development steps. 
To keep the tutorial clean, we implement only simple tests that show the
functionality of {\tt CarRental}'s business logic.
In practice, these white box tests must be much more substantial.

\begin{Example}
\begin{minifbox}
\begin{small}
\begin{verbatim}
    {
      CCM_Local::BigBusiness::Customer person;
      person.id = 2;
      person.first_name = "Thomas";
      person.last_name = "Bernhard";
      ctx->get_connection_maintenance_mirror()->
           createCustomer(person);

      person.id = 3;
      person.first_name = "Karl";
      person.last_name = "Kraus";
      ctx->get_connection_maintenance_mirror()->
           createCustomer(person);
    }
\end{verbatim}
\end{small}
\end{minifbox}
\caption{{\tt retrieveAllCustomers} test case}
\label{example:retrieveAllCustomersTest}
\end{Example}

To test the {\tt retrieveAllCustomers} method, we add two other customers before
we call the method. All retrieved records are compared with their initial values
to validate the query results. 

\begin{Example}
\begin{minifbox}
\begin{small}
\begin{verbatim}
    {
      CCM_Local::BigBusiness::CustomerList person_list;
      person_list = ctx->get_connection_maintenance_mirror()->
                         retrieveAllCustomers();
      assert(person_list.at(2).id == 3);
      assert(person_list.at(2).first_name == "Karl");
      assert(person_list.at(2).last_name == "Kraus");
      assert(person_list.at(1).id == 2);
      assert(person_list.at(1).first_name == "Thomas");
      assert(person_list.at(1).last_name == "Bernhard");
      assert(person_list.at(0).id == 1);
      assert(person_list.at(0).first_name == "Franz");
      assert(person_list.at(0).last_name == "Kafka");
    }      
\end{verbatim}
\end{small}
\end{minifbox}
\caption{{\tt retrieveAllCustomers} test case}
\label{example:retrieveAllCustomersTest}
\end{Example}

Note that the whole example is also available in the 
{\tt test/CppGenerator/tutorial} directory of the {\tt ccmtools} package. 


\newpage
The method's implementation creates a new {\tt CustomerList} which
is returned by value.
In the case of an empty list the {\tt NoCustomerException} is thrown.

\begin{Example}
\begin{minifbox}
\begin{small}
\begin{verbatim}
CustomerList maintenance_impl::retrieveAllCustomers (  )
  throw (LocalComponents::CCMException, NoCustomerException )
{
  if(component->CustomerDB.size() == 0)
      throw NoCustomerException();
  CCM_Local::BigBusiness::CustomerList customer_list;
  std::vector<CCM_Local::BigBusiness::Customer>::iterator pos;
  for(pos = component->CustomerDB.begin(); 
      pos != component->CustomerDB.end(); ++pos) {
    customer_list.push_back(*pos);
  }
  return customer_list;
}
\end{verbatim}
\end{small}
\end{minifbox}
\caption{{\tt retrieveAllCustomers} implementation }
\label{example:retrieveAllCustomersImpl}
\end{Example}


Next, we implement a test case for {\tt updateCustomer}.
A new {\tt Customer} record is created and used as parameter.
To check the update procedure, we use {\tt retrieveCustomer} 
and compare the result record with the new values.
\begin{Example}
\begin{minifbox}
\begin{small}
\begin{verbatim}
    {
      CCM_Local::BigBusiness::Customer person;
      person.id = 1;
      person.first_name = "Werner";
      person.last_name = "Schwab";
      double mileage = 0.0;
      ctx->get_connection_maintenance_mirror()->
           updateCustomer(person);      

      CCM_Local::BigBusiness::Customer another_person;
      another_person = 
        ctx->get_connection_maintenance_mirror()->
             retrieveCustomer(person.id);
      assert(another_person.id == 1);
      assert(another_person.first_name == "Werner");
      assert(another_person.last_name == "Schwab");
    }
\end{verbatim}
\end{small}
\end{minifbox}
\caption{{\tt updateCustomer} test case}
\label{example:}
\end{Example}


\newpage
To implement {\tt updateCustomers} we simply iterate through the {\tt Customer}
vector and compare each item with the given person.
If person matches, we overwrite the vector item and return.
If person does not match, {\tt NoCustomerException} is thrown.

\begin{Example}
\begin{minifbox}
\begin{small}
\begin{verbatim}
void maintenance_impl::updateCustomer ( const Customer& person )
  throw (LocalComponents::CCMException, NoCustomerException )
{
  std::vector<CCM_Local::BigBusiness::Customer>::iterator pos;
  for(pos = component->CustomerDB.begin(); 
      pos != component->CustomerDB.end(); ++pos) {
    if(pos->id == person.id) {
      *pos = person;
      return;
    }
  }
  throw NoCustomerException();  
}
\end{verbatim}
\end{small}
\end{minifbox}
\caption{{\tt updateCustomer} implementation}
\label{example:}
\end{Example}




In the following test case, we delete a {\tt Customer} and try to retrieve the 
deleted record.
If this works, we break the test ({\tt assert(false)}) because this would be an
incorrect behavior. 
Instead, the {\tt RemoveCustomerException} must be thrown.
\begin{Example}
\begin{minifbox}
\begin{small}
\begin{verbatim}
    {
      long id = 1;
      ctx->get_connection_maintenance_mirror()->
           deleteCustomer(id);
      CCM_Local::BigBusiness::Customer person;
      person = ctx->get_connection_maintenance_mirror()->
                    retrieveCustomer(id);
      assert(false); // Customer found => error
    }
\end{verbatim}
\end{small}
\end{minifbox}
\caption{{\tt updateCustomer} test case}
\label{example:}
\end{Example}


\begin{Example}
\begin{minifbox}
\begin{small}
\begin{verbatim}
void maintenance_impl::deleteCustomer ( const long id )
  throw (LocalComponents::CCMException, RemoveCustomerException)
{
  std::vector<CCM_Local::BigBusiness::Customer>::iterator pos;
  for(pos = component->CustomerDB.begin(); 
      pos != component->CustomerDB.end(); ++pos) {
    if(pos->id == id) {
      component->CustomerDB.erase(pos);
      return;
    }
  }
  throw RemoveCustomerException();  
}
\end{verbatim}
\end{small}
\end{minifbox}
\caption{{\tt updateCustomer} implementation}
\label{example:updateCustomer}
\end{Example}

In example \ref{example:updateCustomer}, we iterate through 
the {\tt Customer} vector and compare 
each item with the given id.
If the given record is found, we will remove it from the {\tt Customer} vector.
If the given id can't be found in the{\tt Customer} vector, 
{\tt RemoveCustomerException} is thrown.
In conclusion, we have implemented the whole {\tt maintenance} facet hand in 
hand with its test cases. 

\vspace{3mm}
A great benefit of test driven development is that testing is not longer a 
separated part at the end of a software development process.
In practice, most projects are running out of time before testing can start. 
Thus, lots of functionality is implemented but test cases are skipped 
frequently.
With a test driven approach, one can scale the functionality of a project when 
time is running out, but ensure software quality by having automatic tests.

Another benefit of automatic tests appear in context of code refactoring. 
After making some changes in a component's implementation, we can run all test 
cases to ensure that its functionality is still available.


\vspace{1cm}
In order to implement the second facet, we start with a new test case in 
$\overline{C}$.
For {\tt business} facet tests, we create a new try/catch block that catches
only {\tt NoCustomerException}s.
 
The test adds some miles to an existing customer, retrieves the current miles 
value from the same customer and verifies their equality. 
\begin{Example}
\begin{minifbox}
\begin{small}
\begin{verbatim}
  try {
    {
      long id = 2;
      double miles = 197.7;
      ctx->get_connection_business_mirror()->
           addCustomerMiles(id, miles); 

      double other_miles;
      other_miles = 
        ctx->get_connection_business_mirror()->
             getCustomerMiles(id); 
      assert( abs(other_miles - miles) < 0.001);
    }
  }
  catch(CCM_Local::BigBusiness::NoCustomerException) {
    cerr << "MAINTENANCE ERROR: no customer found!" << endl;
    assert(false);
  }
\end{verbatim}
\end{small}
\end{minifbox}
\caption{{\tt addCustomerMiles} and {\tt getCustomerMiles} test case}
\label{example:}
\end{Example}


\newpage
To make the test run, we have to implement two facet methods.
{\tt addCustomerMiles} iterates through the {\tt Customer} vector
and compares the id values.
If the given {\tt Customer} id is found, the miles value is added to
its current status.
If the given id does not match, a {\tt NoCustomerException} is thrown.
\begin{Example}
\begin{minifbox}
\begin{small}
\begin{verbatim}
void business_impl::addCustomerMiles(const long id, 
                                     const double miles)
  throw (LocalComponents::CCMException, NoCustomerException )
{
  std::vector<CCM_Local::BigBusiness::Customer>::iterator pos;
  for(pos = component->CustomerDB.begin(); 
      pos != component->CustomerDB.end(); ++pos) {
    if(pos->id == id) {
      pos->mileage += miles;
      return;
    }
  }
  throw NoCustomerException();    
}
\end{verbatim}
\end{small}
\end{minifbox}
\caption{{\tt addCustomerMiles} implementation}
\label{example:}
\end{Example}


After adding miles to an existing {\tt Customer} we have to implement a method
that get the current value of a {\tt Customer}'s mileage member.

\begin{Example}
\begin{minifbox}
\begin{small}
\begin{verbatim}
double business_impl::getCustomerMiles ( const long id )
  throw (LocalComponents::CCMException, NoCustomerException )
{
  std::vector<CCM_Local::BigBusiness::Customer>::iterator pos;
  for(pos = component->CustomerDB.begin(); 
      pos != component->CustomerDB.end(); ++pos) {
    if(pos->id == id) {
      return pos->mileage;
    }
  }
  throw NoCustomerException(); 
}
\end{verbatim}
\end{small}
\end{minifbox}
\caption{{\tt getCustomerMiles} implementation}
\label{example:}
\end{Example}

Again, we iterate through the {\tt Customer} vector and compare the id values.
If we find a {\tt Customer} with the right id, we read its mileage value and
return it to the caller.
In the other case, as usual, the {\tt NoCustomerException} is thrown.



\newpage
The following test case sets the facet's {\tt dollars\_per\_mile} attribute
and retrieves the current mileage and Dollar status of a particular 
{\tt Customer} determined by its id.
Thus, the component's calculation can be verified.

\begin{Example}
\begin{minifbox}
\begin{small}
\begin{verbatim}
    {
      long id = 2;
      double dollars, miles;
      double factor = 5.3;
      ctx->get_connection_business_mirror()->
           dollars_per_mile(factor);
      miles = ctx->get_connection_business_mirror()->
                   getCustomerMiles(id);
      dollars = ctx->get_connection_business_mirror()->
                     getCustomerDollars(id); 
      assert( abs(dollars - miles*factor) < 0.001);
    }
\end{verbatim}
\end{small}
\end{minifbox}
\caption{{\tt dollars\_per\_mile} and {\tt getCustomerDollars} test case}
\label{example:}
\end{Example}


\begin{Example}
\begin{minifbox}
\begin{small}
\begin{verbatim}
double business_impl::dollars_per_mile()
  throw ( LocalComponents::CCMException)
{
  return _dollars_per_mile;
}

void business_impl::dollars_per_mile(const double value )
  throw ( LocalComponents::CCMException)
{
  _dollars_per_mile = value;
}
\end{verbatim}
\end{small}
\end{minifbox}
\caption{{\tt dollars\_per\_mile} attribute implementation}
\label{example:get-set}
\end{Example}

Example \ref{example:get-set} shows the implementation of IDL attributes within 
a facet.
Note that in the corresponding header file the {\tt business\_impl} class 
declaration must be extended to provide a private member called 
{\tt \_dollars\_per\_mile}. 

\begin{Example}
\begin{minifbox}
\begin{small}
\begin{verbatim}
double business_impl::getCustomerDollars ( const long id )
  throw (LocalComponents::CCMException, NoCustomerException )
{
  return getCustomerMiles(id) * _dollars_per_mile;
}
\end{verbatim}
\end{small}
\end{minifbox}
\caption{{\tt getCustomerDollars()} implementation}
\label{example:getCustomerDollars}
\end{Example}

Life would be pretty ease if all business logic would be as simple to implement 
as \\ 
{\tt getCustomerDollars} shown in example \ref{example:getCustomerDollars} 
;-)


\newpage
Finally, the last test case resets a {\tt Customer}'s mileage value and 
retrieves its Dollar status. Of course, it has to be null.

\begin{Example}
\begin{minifbox}
\begin{small}
\begin{verbatim}
    {
      long id = 2;
      double dollars;
      ctx->get_connection_business_mirror()->
           resetCustomerMiles(id);
      dollars = 
        ctx->get_connection_business_mirror()->
             getCustomerDollars(id); 
      assert( abs(dollars) < 0.001);
    }
\end{verbatim}
\end{small}
\end{minifbox}
\caption{{\tt resetCustomerMiles} test case}
\label{example:}
\end{Example}



\begin{Example}
\begin{minifbox}
\begin{small}
\begin{verbatim}
void business_impl::resetCustomerMiles ( const long id )
  throw (LocalComponents::CCMException, NoCustomerException )
{
  std::vector<CCM_Local::BigBusiness::Customer>::iterator pos;
  for(pos = component->CustomerDB.begin(); 
      pos != component->CustomerDB.end(); ++pos) {
    if(pos->id == id) {
      pos->mileage = 0.0;
      return;
    }
  }
  throw NoCustomerException(); 
}
\end{verbatim}
\end{small}
\end{minifbox}
\caption{{\tt resetCustomerMiles} implementation}
\label{example:resetCustomer}
\end{Example}
As shown in example \ref{example:resetCustomer}, {\tt resetCustomerMiles} 
searches in the {\tt Customer}
vector for the given id and set the mileage value to null.
If no {\tt Customer} id matches, the {\tt NoCustomerException} is thrown.

Well done! You have implemented the first component using the CCM Tools.


\newpage
%------------------------------------------------------------------------------
\subsection{Deploy the component}
%------------------------------------------------------------------------------

Deploying a component means that the component's libraries and header files 
will be stored in a component repository, which is simply a directory on your 
computer (specified by {\tt PREFIX} in {\tt ~/.confix}).

Component deployment is performed by Confix:
\begin{small}
\begin{verbatim}
~/example1> confix.py --packageroot=`pwd`/CarRental/component \
                      --make --targets=install
\end{verbatim}
\end{small}

After installing a component, the component repository contains the following
file structures:
\begin{small}
\begin{verbatim}
include/
|-- CCM_Local
|   |-- BigBusiness
|   |   |-- CCM_Session_CarRental/
|   |   |-- CCM_Session_CarRental_mirror/
|   |   |-- CreateCustomerException.h
|   |   |-- Customer.h
|   |   |-- CustomerBusiness.h
|   |   |-- CustomerList.h
|   |   |-- CustomerMaintenance.h
|   |   |-- NoCustomerException.h
|   |   `-- RemoveCustomerException.h
\end{verbatim}
\end{small}

\begin{small}
\begin{verbatim}
lib/
|-- libCarRental_CCM_Local_BigBusiness.a
|-- libCarRental_CCM_Local_BigBusiness_CCM_Session_CarRental.a
|-- libCarRental_CCM_Local_BigBusiness_CCM_Session_CarRental_mirror.a
|-- libCarRental_CCM_Local_BigBusiness_CCM_Session_CarRental_mirror_share.a
|-- libCarRental_CCM_Local_BigBusiness_CCM_Session_CarRental_share.a
`-- libCarRental_src.a
\end{verbatim}
\end{small}




%------------------------------------------------------------------------------
\subsection{Undeploy the component}
%------------------------------------------------------------------------------

To remove the component from the component repository, type:

\begin{small}
\begin{verbatim}
~/example1>  confix.py --packageroot=`pwd`/CarRental/component \
                       --make --targets="uninstall" 
\end{verbatim}
\end{small}

Installing is not necessary during component development. 
For implementing test cases and business logic we only need the source and
build directory (specified by {\tt BUILDROOT} in {\tt ~/.confix}).
 
Note that we have to deploy our component again to use it from a client,
as described in the following section. 

%\newpage
%------------------------------------------------------------------------------
%\subsection{Component packaging}
%------------------------------------------------------------------------------
%\dots


\newpage
%------------------------------------------------------------------------------
\subsection{Write a component client}
%------------------------------------------------------------------------------

A component based application builds up component assemblies using deployed 
components. In this section we will show this using a simple client that
instantiates the {\tt CarRental} component and calls the component's operations.

First, we will add a {\tt CatRentalClient} directory to our project and implement
the client code in the subdirectory {\tt src}:
\begin{small}
\begin{verbatim}
example1/
|-- CarRental/
`-- CarRentalClient
    `-- src
        |-- Makefile.py
        `-- client.cc
\end{verbatim}
\end{small}

The client needs a bunch of included header files that are generated by the CCM Tools.
There are different namespaces defined in the header files that we are using in the 
client code.
\begin{Example}
\begin{minifbox}
\begin{small}
\begin{verbatim}
#include <LocalComponents/CCM.h>
#include <CCM_Local/HomeFinder.h>
#include <WX/Utils/debug.h>
#include <WX/Utils/smartptr.h>

#include <CCM_Local/BigBusiness/CCM_Session_CarRental/ \
          CarRentalHome_gen.h>
#include <CCM_Local/BigBusiness/CCM_Session_CarRental/ \
          CarRental_gen.h>

using namespace std;
using namespace WX::Utils;
using namespace CCM_Local;
using namespace BigBusiness;
using namespace CCM_Session_CarRental;
\end{verbatim}
\end{small}
\end{minifbox}
\caption{Client's include and namespace section.}
\label{example:}
\end{Example}

In this simple client, the main function contains all the necessary code to use
a local component. 
\begin{itemize}
\item Client's bootstrap and tear--down code (Example \ref{example:ClientBootsrap}).\\
Before a component can be used, a component home must be instantiated and registered
by the {\tt HomeFinder}.
After using a component, the component home instance must be removed from memory and
from the {\tt HomeFinder}'s list.
\item Component's instantiation code (Example \ref{example:ClientComponentInstance}). \\
A client gets a home reference from the {\tt HomeFinder}, creates a component 
instance and asks for the component's facets.
To denote the end of this instantiation phase, the client calls {\tt configuration\_complete()}.
After all business calls, the client has to remove its component instance.

\item Client's application code (Example \ref{example:ClientApplication}). \\
Here the client calls some component operations to make its business work.
We have implemented a simple use case known from our mirror component test cases.
\end{itemize}

\begin{Example}
\begin{minifbox}
\begin{small}
\begin{verbatim}
int main(int argc, char *argv[])
{
  int error = 0;
  LocalComponents::HomeFinder* homeFinder;
  homeFinder = HomeFinder::Instance (  );

  error = local_deploy_CarRentalHome("CarRentalHome");
  if(error) {
    cerr << "BOOTSTRAP ERROR: Can't deploy component homes!" 
         << endl;
    return(error);
  }

  // Component instantiation code

  error += local_undeploy_CarRentalHome("CarRentalHome");
  if(error) {
    cerr << "TEARDOWN ERROR: Can't undeploy component homes!" 
         << endl;
    return error;
  }
}
\end{verbatim}
\end{small}
\end{minifbox}
\caption{Client's bootstrap and tear--down code.}
\label{example:ClientBootsrap}
\end{Example}

\begin{Example}
\begin{minifbox}
\begin{small}
\begin{verbatim}
  try {
    SmartPtr<CarRentalHome> 
      myCarRentalHome(dynamic_cast<CarRentalHome*>
      (homeFinder->find_home_by_name("CarRentalHome").ptr()));

    SmartPtr<CarRental> myCarRental = myCarRentalHome->create();

    SmartPtr<CCM_Local::BigBusiness::CustomerMaintenance> 
      maintenance = myCarRental->provide_maintenance();    
    SmartPtr<CCM_Local::BigBusiness::CustomerBusiness> 
      business = myCarRental->provide_business();

    myCarRental->configuration_complete();

    // Client's application code.

    myCarRental->remove();
  } 
  catch ( ... )  {
    cout << "Client: there is something wrong!" << endl;
    error = -1;
  }
\end{verbatim}
\end{small}
\end{minifbox}
\caption{Component's instantiation code.}
\label{example:ClientComponentInstance}
\end{Example}



\begin{Example}
\begin{minifbox}
\begin{small}
\begin{verbatim}
    business->dollars_per_mile(5.5);
    
    CCM_Local::BigBusiness::Customer person;
    person.id = 1;
    person.first_name = "Franz";
    person.last_name = "Kafka";
    maintenance->createCustomer(person);

    business->addCustomerMiles(1, 120.0); 
    person = maintenance->retrieveCustomer(1);
    double dollars = business->getCustomerDollars(1); 

    cout << " Customer: " << person.first_name 
         << " " << person.last_name << endl;
    cout << " Miles: " <<  person.mileage << endl;
    cout << " to pay: " << dollars << " Dollars" << endl;
\end{verbatim}
\end{small}
\end{minifbox}
\caption{Client's application code.}
\label{example:ClientApplication}
\end{Example}

To build and run the client application, we use the following statements:
\begin{small}
\begin{verbatim}
~/example1> confix.py --advanced --bootstrap --configure \
                      --packageroot=`pwd`/CarRentalClient \
                      --make --targets="install"

~/example1> <.confix/'PREFIX'>/bin/CarRentalClient_src_client
\end{verbatim}
\end{small}


The advantage here is that development of local components is separated from
client development. 
While components can be implemented in independently by different developers, an
application (or client) can use deployed components from a repository.

