% $Id$

%==============================================================================
\chapter{First Components}
%==============================================================================


%------------------------------------------------------------------------------
\section{Overview}
%------------------------------------------------------------------------------

This section shows how to implement a simple hello world like component using
the {\it CCM Tools}.
Of course, this is not an application and a hello world component is
a structural overkill, but the example is well suited to demonstrate the use
of our tool set.

\begin{figure}[htbp]
    \begin{center}
        \includegraphics [width=10cm,angle=0] {DevelopmentProcess}
        \caption{Component development process}
        \label{DevelopmentProcess}
    \end{center}
\end{figure}

As shown in Fig.~\ref{DevelopmentProcess},
the development process defines different roles that reflect different abstraction
levels of components.
\begin{description}
\item [Component Designer] specifies the structure of the component by defining
an {\it Interface Description Language} (IDL) File. This IDL interfaces are only a 
syntax definition and do not describe the behavior of the component.
\item [Component Developer] implements the behavior of the component in
respect to the defined interfaces. 
\end{description}

The {\it CCM Tools} supports the development process by providing an IDL parser
and different code generators.
Generated code is used in both parts of a component:
\begin{description}
\item [Component Logic] is the implementation of the component's structure.
It contains mechanisms to create and destroy component instances as well as access
and navigation methods for ports and attributes.
This code can be completely generated from the IDL description. 
\item [Business Logic] is the implementation of the component's behavior in respect 
to the application domain. 
Generator tools can create skeleton code, the component developer, however, 
implements the business code without care about the overall structure of the application.
\end{description}

Finally, we use the {\it Confix} build tool to create a library from the generated and
hand-crafted source code. The component libraries are stored in a component repository on the
server side and are used to assemble modular applications.

  

%------------------------------------------------------------------------------
\section{The designer's job}
%------------------------------------------------------------------------------

Let's start to implement the first component using the {\it CCM Tools}.
We define an IDL file named {\tt hello.idl} that describes the component's structure
and is stored in a {\tt hello} directory:
\begin{verbatim}
        interface Console {
          long println(in string s2);
        };

        component Hello {
          attribute string prompt;	
          provides Console console;
        };

        home HelloHome manages Hello {
        };
\end{verbatim}

The IDL file  defines a {\tt Console} interface that contains a method which prints 
out a string and returns the number of printed characters. 
The {\tt Hello} component provides the {\tt console} interface as a facet 
and an attribute called {\tt prompt}.
Finally, the {\tt HelloHome} is defined to manage the {\tt Hello} component.
Thus, we have created the following file structure:  
\begin{verbatim}
        hello/
        |-- hello.idl
\end{verbatim}

Well done, we have designed an example component and also created the corresponding
IDL definition. 
Note that the designer defines the external view of a component as a part of the
whole application structure.



%------------------------------------------------------------------------------
\section{The developer's job}
%------------------------------------------------------------------------------

Remember, the component developer implements the behavior of the component in
respect to the defined interfaces. This section shows how the {\it CCM Tools}
make this job easier.

%------------------------------------------------------------------------------
\subsection{Set up component environment}
%------------------------------------------------------------------------------

Before we can develop components, we have to create the component's environment.
First, we create or extend the {\it Confix} configuration file {\tt ~/.confix}
(see the {\tt Confix} manual for more details):
\begin{verbatim}
ccmtools_profile = {
    'PREFIX': '<install-path>/ccmTools',
    'BUILDROOT': '/tmp',
    'ADVANCED': 'true',
    'CONFIX': {},
    'CONFIGURE': {
       'ENV': {},
       'ARGS': []
    },
}
    
PROFILES = {
    'ccmtools': ccmtools_profile,
    #...
}
\end{verbatim}

\noindent
Second, we install the local C++ component runtime libraries:
\begin{verbatim}
~/hello> ccmtools-c++-environment
\end{verbatim}

\noindent
That's it, now we are ready to implement components.
Note that the component's environment is installed only once for all 
components we implement.


%------------------------------------------------------------------------------
\subsection{Create an empty component}
%------------------------------------------------------------------------------

To get the component logic and the skeletons of the business logic from the IDL file, 
we start the local C++ code generator:
\begin{verbatim}
~/hello> ccmtools-c++-generate -c 1.0 -d -p hello1.0 hello.idl
\end{verbatim}
The {\tt ccmtools-c++-generate} call accepts a bunch of command line parameters. In the
example we only use a few to demonstrate the possibilities.
An in depth description of the {\it CCM Tools} goes out of this tutorial but 
here is a short description of the used options:
\begin{itemize}
\item {\tt -c VERSION, --code-version=VERSION }\\
Set VERSION of generated code, the default component version is ``0.1''.
Note that the component's version number can be asked at runtime.  

\item {\tt -d, --debug }\\
Enable debugging in generated code. The generated code produces a lot of
debug messages that can be used to trace the program execution.

\item {\tt -p NAME, --package=NAME}\\
Set package name to NAME, the default package name is ``ccmtools-package''.
Note that the package name is used for the name of the generated subdirectory.

\item {\tt -h, --help}\\
Prints out a short description of the available command line parameter.
\end{itemize}
After calling {\tt ccmtools-c++-generate}, we have a new subdirectory ({\tt hello1.0}) that 
contains the generated component logic. Also, the skeletons for the business logic are
created in the current directory:
\begin{verbatim}
  hello/
  |-- HelloHome_app.cc
  |-- HelloHome_app.h
  |-- HelloHome_mirror_app.cc
  |-- HelloHome_mirror_app.h
  |-- Hello_app.cc
  |-- Hello_app.h
  |-- Hello_mirror_app.cc
  |-- Hello_mirror_app.h
  |-- hello.idl
  |-- hello_user_types.h
  |-- hello1.0/
\end{verbatim}
The component's developer does not care about the generated component logic in the
subdirectory. The interesting files are in the current directory as described below:
\begin{itemize}
\item {\tt HelloHome\_app.[h,cc]} \\
Contains the factory methods for component creation. 
Note that the default {\tt create()} method is already implemented.
\item {\tt Hello\_app.[h,cc]} \\
Contains the component's business logic skeletons.
\item {\tt hello\_user\_types.h} \\
Contains the default IDL to C++ mappings. Note that these mappings can be 
changed by the user.
\end{itemize}
The generated business logic skeletons contains enough code to be compilable.
Each method includes a debug statement that prints out the file name and the
method name of the executed function.
Thus, we have an empty but compilable component - let's compile it!
\begin{verbatim}
~/hello> ccmtools-c++-make -p hello1.0
\end{verbatim}

\noindent
The {\tt ccmtools-c++-make} script also accepts 
command line parameters as described below:
\begin{itemize}
\item {\tt -p NAME, --package=NAME}\\
Use package named NAME, the default package name is ``ccmtools-package''.

\item {\tt -h, --help}\\
Prints out a short description of the available command line parameter.
\end{itemize}

After compiling the source code, a test client calls all operations of
the generated component logic.
The running test prints out a lot of debugging messages
that documents the execution flow of the program - the first component is running!


%------------------------------------------------------------------------------
\subsection{Test driven development}
%------------------------------------------------------------------------------

Software development is an iterative process. Kent Beck proposed a 
{\it Test Driven Development} (TDD) \cite {Beck2003TDD} that starts by implementing 
the test - before implementing the application!

\begin{figure}[htbp]
    \begin{center}
        \includegraphics [width=6cm,angle=0] {TestDrivenDevelopment}
        \caption{Test driven development process}
        \label{DevelopmentProcess}
    \end{center}
\end{figure}

As shown in Fig.~\ref{DevelopmentProcess}, we use the TDD idea in context of components.
The {\it CCM Tools} generate for every component $C$ a mirror component $\overline{C}$.
Each input port of $C$ corresponds to a output port in $\overline{C}$ and vice versa.
A test client coordinates the creating and connecting of the components as well as the
test calls to the facets of $C$.  

While the test client is part of the generated code in the {\tt hello1.0} directory,
the current component directory contains the skeletons of the mirror component business
logic:
\begin{itemize}
\item {\tt HelloHome\_mirror\_app.[h,cc]} \\
Contains the factory methods for mirror component creation.
Note that the default {\tt create()} method is already implemented.
\item {\tt Hello\_mirror\_app.[h,cc]} \\
Contains the skeletons of the mirror component's test code.
\end{itemize}

\noindent
Note that the mirror component $\overline{C}$ and the test client are always generated 
at the same time as the component itself - without additional development effort.

\noindent
The component's unit test, that creates a $C$ and a 
$\overline{C}$ instance and connects these components together as shown in 
Fig.~\ref{DevelopmentProcess}, starts after compiling or if we type:
\begin{small}
\begin{verbatim}
~/hello> /tmp/hello1.0/CCM_Test/hello1.0_CCM_Test__check_CCM_Session_Hello
\end{verbatim}
\end{small} 



%------------------------------------------------------------------------------
\subsection{Write the test first}
%------------------------------------------------------------------------------

That's a good time to start the development of business code.
First, we have to define the behavior of the business code in an executable
semantic.
Thus we write a test case in the mirror component ({\tt Hello\_mirror\_app.cc}):
\begin{verbatim}
void
CCM_Hello_mirror_impl::ccm_activate (  )
  throw ( localComponents::CCMException )
{
  DEBUGNL ( " CCM_Hello_mirror_impl->ccm_activate (  )" );

  const string s = "Hello World";
  cout << "Start of Hello unit test:" << endl;
  int number_of_char =
    ctx->get_connection_console_mirror().ptr()->println(s);
  assert(number_of_char == s.length());	
}
\end{verbatim}

\noindent
The test case is implemented in the {\tt ccm\_activate()} callback method of the mirror
component because this method is called after creating and connecting
$C$ and $\overline{C}$.  
We use the context object {\tt ctx} to get a smart pointer to the facet of $C$
before we call the {\tt println()} operation.
To compile and run the unit test again, we type:
\begin{verbatim}
~/hello> ccmtools-c++-make -p hello1.0
\end{verbatim}
From the output of the test client we can see that the {\tt println()} method
of $C$ is called from the mirror component $\overline{C}$ but {\tt println()}
the test fails - remember, there is no business code implemented in {\tt println()}!




%------------------------------------------------------------------------------
\subsection{Write the business logic}
%------------------------------------------------------------------------------

Finally, we can implement the business logic that satisfies the written test case.
We open the {\tt Hello\_app.cc} file and add a single line:
\begin{verbatim}
long
console_impl::println ( const std::string& s2 )
{
  DEBUGNL ( " console_impl->println ( s2 )" );
  cout << s2 << endl;
  return s2.length();
}
\end{verbatim}

\noindent
The {\tt println()} method simply prints out the given string and returns the number
of characters of the string - that's it!
We run the test again to see if it's working:
\begin{verbatim}
~/hello> ccmtools-c++-make -p hello1.0
\end{verbatim}

\noindent
We are lucky, the unit test passed!

\noindent
Note that all the hand crafted code is in the current directory, there is no
need to touch the generated code in {\tt hello1.0}.
To protect the business code from overwriting, the {\tt ccmtools-c++-generate} 
tool creates the {\tt *\_app.[h,cc]} files only if they do not exist. 


%------------------------------------------------------------------------------
\subsection{Deploy the component}
%------------------------------------------------------------------------------

To deploy a component means that the component's source code will be compiled
to a component library file that will be stored in the component repository.
The {\it CCM Tools} provide this functionality in a single call:
\begin{verbatim}
~/hello> ccmtools-c++-install -p hello1.0
\end{verbatim}
The {\tt ccmtools-c++-install} script accepts the following command line parameters:
\begin{itemize}
\item {\tt -p NAME, --package=NAME}\\
Use package named NAME, the default package name is ``ccmtools-package''.

\item {\tt -h, --help}\\
Prints out a short description of the available command line parameter.
\end{itemize}

\noindent
At this point of time, the component is part of the component repository 
and can be used in a component assembly or a local client.


%------------------------------------------------------------------------------
\subsection{Write a local test client}
%------------------------------------------------------------------------------

A component based application builds up component assemblies using
already deployed components. 
We show this task on a simple test client that instantiates the {\tt Hello} 
component and calls the component's operations.

First, we add a {\tt test} directory to our project and
implement the client code in the subdirectory {\tt client}. 
\begin{verbatim}
    hello/
    |-- hello1.0/
    |-- test/
    |   |-- client/
    |   |   |-- client.cc 
\end{verbatim}

\noindent
The test client needs a bunch of included header files that are generated by 
the {\it CCM Tools}. There are different namespaces defined in the header files
that we are using in the client code.
\begin{small}
\begin{verbatim}
#include <localComponents/CCM.h>    
#include <CCM_Local/HomeFinder.h>
#include <CCM_Utils/Debug.h>

#include <CCM_Local/CCM_Session_Hello/HelloHome_gen.h>
#include <CCM_Local/CCM_Session_Hello/Hello_gen.h>

using namespace std;
using namespace CCM_Utils;
using namespace CCM_Local;
using namespace CCM_Session_Hello;
\end{verbatim}
\end{small}

\noindent
The main function contains the whole code that is necessary to use a 
local component.
At the local home finder, we register the {\tt HelloHome} instance.
\begin{small}
\begin{verbatim}
int main ( int argc, char *argv[] )
{
  // Set debugging mode
  Debug::set_global(true);

  // Get in instance of the local HomeFinder and register component homes
  localComponents::HomeFinder* homeFinder = HomeFinder::Instance (  );
  try {                       
    homeFinder->register_home(create_HelloHomeAdapter(), "HelloHome" );
  } catch ( ... )  {
    cout << "Aut'sch: when registering homes!" << endl;
    return -1;
  }
\end{verbatim}
\end{small}

\noindent
After home registration, we can find the home by its name and create a
component instance. From the component instance we get a reference
to the {\tt Console} facet.
To indicate the component the end of create and configuration, we call 
{\tt configuration\_complete} method.
\begin{small}
\begin{verbatim}
  try {
    // Find component home
    SmartPtr<HelloHome> myHelloHome (dynamic_cast<HelloHome*>
      ( homeFinder->find_home_by_name ( "HelloHome").ptr ()));

    // Create a component instance
    SmartPtr<Hello> myHello = myHelloHome.ptr()->create();
  
    // Get a facet reference
    SmartPtr<CCM_Console> console = myHello.ptr()->provide_console();

    // Configure the component's attribute
    myHello.ptr()->prompt("=->");

    // Component configuration finished
    myHello.ptr()->configuration_complete();
\end{verbatim}
\end{small}

\noindent
Now we are ready to use the operations provided by the component and its
facet.
\begin{small}
\begin{verbatim}    
    // Call operations on the component and its facet
    cout << "Version = " << myHello.ptr()->getComponentVersion() << endl;
    cout << "Date = " << myHello.ptr()->getComponentDate() << endl;
	
    console.ptr()->println("Hello from test client");
\end{verbatim}
\end{small}

\noindent
To tear the application town, we remove the component instance from memory
and unregister the component's home.
\begin{small}	
\begin{verbatim}      
    // Destroy component instance
    myHello.ptr()->remove();

    // Unregister component home
    homeFinder->unregister_home("HelloHome");
  }
  catch ( localComponents::HomeNotFound ) {
    cout << "Aut'sch: can't find a home!" << endl;
    return -1;
  }
  catch ( ... )  {
    cout << "Aut'sch: there is something wrong!" << endl;
    return -1;
  }
  return 0;
}
\end{verbatim}
\end{small}


\noindent
To compile and install the local test client we use the {\tt Confix} tool
that is also used by the component build process.
Note that we use the same confix profile as we use by the {\it CCM Tools}.
\begin{verbatim}
~/hello/test> confix.py --bootstrap --configure  \
                        --make --targets=install --profile=ccmtools
\end{verbatim}

\noindent
Finally, we can run the test client:
\begin{verbatim}
~/hello> test_client_client
\end{verbatim}
The advantage here is that the development of the 
local component is separated from the development of the test client.
Thus, a deployed component can be used in different applications at the same time.


%------------------------------------------------------------------------------
\subsection{Integrate existing business logic}
%------------------------------------------------------------------------------

The component's business code, of the example above, is very simple and therefore
direct implemented in the {\tt Hello\_app.[h,cc]} files.
Usually, the business code has been already developed in a separated 
{\tt business} directory or is some kind of legacy code: 
\begin{verbatim}
    hello/
    |-- hello1.0/
    |-- test/
    |   |-- client/
    |   |-- business/
    |   |   |-- legacy.h
    |   |   |-- legacy.cc
\end{verbatim}

\noindent
The implemented business logic is a legacy code that provides a 
{\tt print\_string} method.
\begin{verbatim}
    // legacy.h	
    #include<stdio.h>
    extern void print_string(const char* s);

    // legacy.cc
    #include "legacy.h"
    void print_string(const char* s)
    {
      printf(s);
    }
\end{verbatim}

\noindent
We propose that the business logic is build in the same way as the test client,
using the {\it Confix} tool:
\begin{verbatim}
test> confix.py --bootstrap --configure \
                --make --targets=install --profile=ccmtools
\end{verbatim}

\noindent
After that, the {\tt Hello\_app.cc} file can be modified in the following way
to integrate the separate compiled business code:
\begin{verbatim}
#include<legacy.h>
//...
void
console_impl::println ( const std::string s2 )
{
  DEBUGNL ( " console_impl->println ( s2 )" );
  print_string(s2.c_str());  // call the legacy code
  return s2.length();
}
\end{verbatim}

\noindent
In conclusion, to build the final component and the test client, we type:
\begin{verbatim}
~/hello> ccmtools-c++-make -p hello1.0
~/hello/test> confix.py --bootstrap --configure \
                        --make --targets=install --profile=ccmtools
~/hello/test> test_client_client
\end{verbatim}

\noindent
In summary, we have included legacy code in the component's business logic and 
use the existing client to integrate the component in an application.


%------------------------------------------------------------------------------
\subsection{Undeploy the component}
%------------------------------------------------------------------------------

To remove the component from the component repository, type:
\begin{verbatim}
~/hello> ccmtools-c++-uninstall -p hello1.0
\end{verbatim}
The command line parameters are the same as by {\tt ccmtools-c++-install}.
Note that the source code is not affected by the undeployment process.




%------------------------------------------------------------------------------
\section{Connecting components by facet and receptacle}
%------------------------------------------------------------------------------

The CORBA component model also defines the concept of receptacles that are
connection points for facet references.
This section describes an example that extends the previous example with
a {\tt Display} component which is connected to the {\tt Hello} component.


%------------------------------------------------------------------------------
\subsection{Definition of two components}
%------------------------------------------------------------------------------

We define a new component called {\tt Display} that provides an {\tt LCD}
interface as a facet.
The {\tt Hello} component uses the {\tt LCD} interface as an receptacle.
Note that a facet to receptacle connection can only be established when
both ports handle the same interface.

\begin{verbatim}
    interface LCD {
      void display_text(in string s);
    };

    component Display {
      attribute string prompt;
      provides LCD lcd;
    };

    home DisplayHome manages Display {
    };



    interface Console {
      long println(in string s2);
    };

    component Hello {
      provides Console console;
      uses LCD lcd;
    };

    home HelloHome manages Hello {
    };
\end{verbatim}

\noindent
As long as we define the components in the same file, we have the same file structure
as before:  
\begin{verbatim}
        hello/
        |-- hello.idl
\end{verbatim}

%------------------------------------------------------------------------------
\subsection{Implementation of two components}
%------------------------------------------------------------------------------

We start with generating and building the empty components:
\begin{verbatim}
~/hello> ccmtools-c++-generate -c 1.0 -d -p hello1.0 hello.idl
~/hello> ccmtools-c++-make -p hello1.0
\end{verbatim}

\noindent
For each of the components a mirror component has been generated as well as a
test client that connects every component with its mirror component and calls 
the component's operations - Test-Driven Development.

To make it short, we implemented the following methods in the business logic
of the {\tt Display} and the {\tt Hello} component:
\begin{verbatim}
// Display_app.cc
void
lcd_impl::display_text ( const std::string& s )
{
  DEBUGNL ( " lcd_impl->display_text ( s )" );
  cout << component->prompt() << s;
}

// Hello_app.cc
long
console_impl::println ( const std::string& s2 )
{
  DEBUGNL ( " console_impl->println ( s2 )" );
  component->ctx->get_connection_lcd().ptr()->display_text(s2);
  cout << endl;
  return s2.length();	
}
\end{verbatim}

\noindent
We build the components again and install them in the component repository:
\begin{verbatim}
~/hello> ccmtools-c++-make -p hello1.0
~/hello> ccmtools-c++-install -p hello1.0
\end{verbatim}


%------------------------------------------------------------------------------
\subsection{Write a local test client for two components}
%------------------------------------------------------------------------------
To show the use of both components in an application, we write a test client in the
following file structure:
\begin{verbatim}
    hello/
    |-- hello1.0/
    |-- test/
    |   |-- client/
    |   |   |-- client.cc 
\end{verbatim}

\noindent
The client starts with a lot of include and using namespace statements.
Note that the included header files reflect the used components.
\begin{small}
\begin{verbatim}
#include <localComponents/CCM.h>
#include <CCM_Local/HomeFinder.h>
#include <CCM_Utils/Debug.h>

#include <CCM_Local/CCM_Session_Display/Display_gen.h>
#include <CCM_Local/CCM_Session_Hello/Hello_gen.h>
#include <CCM_Local/CCM_Session_Display/DisplayHome_gen.h>
#include <CCM_Local/CCM_Session_Hello/HelloHome_gen.h>

using namespace std;
using namespace CCM_Utils;
using namespace CCM_Local;
using namespace CCM_Session_Display;
using namespace CCM_Session_Hello;
\end{verbatim}
\end{small}

\noindent
In the main function, after setting the debugging mode, 
both component homes are registered at the home finder.
\begin{small}
\begin{verbatim}
int main ( int argc, char *argv[] )
{
  // Set debugging mode
  Debug::set_global(true);

  // Get in instance of the local HomeFinder and register component homes
  localComponents::HomeFinder* homeFinder = HomeFinder::Instance (  );
  try {                       
    homeFinder->register_home( create_DisplayHomeAdapter(), "DisplayHome" );
    homeFinder->register_home( create_HelloHomeAdapter(), "HelloHome" );
  } catch ( ... )  {
    cout << "Aut'sch: when register homes!" << endl;
    return -1;
  }
\end{verbatim}
\end{small}

\noindent
Now, we can use the home finder to find the homes by name.
We create an instance of each component, get references the the component's
facets and connect the facet of {\tt Display} with the receptacle of {\tt Hello}.
Note that both ports are defined by the same {\tt LCD} interface.
\begin{small}
\begin{verbatim}
  try {
    // Find component home
    SmartPtr<DisplayHome> myDisplayHome (dynamic_cast<DisplayHome*>
      ( homeFinder->find_home_by_name ( "DisplayHome").ptr ()));
    SmartPtr<HelloHome> myHelloHome (dynamic_cast<HelloHome*>
      ( homeFinder->find_home_by_name ( "HelloHome").ptr ()));

    // Create component instance
    SmartPtr<Hello> myHello = myHelloHome.ptr()->create();
    SmartPtr<Display> myDisplay = myDisplayHome.ptr()->create();
 
    // Get facet references and connect facets to receptacles
    SmartPtr<CCM_Console> console = myHello.ptr()->provide_console();	
    SmartPtr<CCM_LCD> lcd = myDisplay.ptr()->provide_lcd();
    myHello.ptr()->connect_lcd(lcd);

    // Configure the component's attribute
    myDisplay.ptr()->prompt("-=> ");

    // Component configuration finished	
    myHello.ptr()->configuration_complete();
    myDisplay.ptr()->configuration_complete();
\end{verbatim}
\end{small}

\noindent
The test client has build an component assembly of two components in memory and
can use its functionality. 
\begin{small}
\begin{verbatim}  
    // Call operations on the component assembly
    cout << "Display Version = " 
         << myDisplay.ptr()->getComponentVersion()  << endl;
    cout << "Hello Version = "  
         << myHello.ptr()->getComponentVersion() << endl;

    console.ptr()->println("Hello from the client");
\end{verbatim}
\end{small}

\noindent
To tear the component assembly town, we disconnect the ports and 
remove the component instances from memory.
We also unregister the component's homes from the home finder.
\begin{small}	
\begin{verbatim}    
    // Disconnect component ports
    myHello.ptr()->disconnect_lcd();    

    // Destroy component instances
    myDisplay.ptr()->remove();
    myHello.ptr()->remove();

    // Unregister component homes
    homeFinder->unregister_home("DisplayHome");
    homeFinder->unregister_home("HelloHome");
  }
  catch ( localComponents::HomeNotFound ) {
    cout << "Aut'sch: can't find a home!" << endl;
    return -1;
  }
  catch ( ... )  {
    cout << "Aut'sch: there is something wrong!" << endl;
    return -1;
  }
  return 0;
}
\end{verbatim}
\end{small}

\noindent
We build the test client with {\tt Confix} and run the binary:
\begin{verbatim}
~/hello/test> confix.py --bootstrap --configure  \
                        --make --targets=install --profile=ccmtools
~/hello/test> test_client_client
\end{verbatim}

\noindent
Isn't it cool?\\
We have run a test client that uses two components which are connected via facet 
and receptacle to build a component assembly.
In the same way, we can create more complex assemblies that form a real
component based application as well.

Look at the printed debugging information to trace the execution thread of the
test client. We can also switch off the messages by setting the debugging mode
in the main function to {\tt false}.