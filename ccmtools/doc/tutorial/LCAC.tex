%==============================================================================
\chapter{Local Component Adapter Concept}
%==============================================================================
\begin{flushright}
{\it }
\end{flushright}

%==============================================================================
\section{Introduction}
%==============================================================================
From practical experiences we learned that there is a need for improvements
in the CCM to be usable. The following sections describe the
new concepts we introduced in respect to the CCM specification.


%==============================================================================
\section{Remote components}
%==============================================================================

The CCM specification describes only remote components where all ports are
accessible from CORBA clients.
Some open source implementations like MicoCCM \cite{MicoCCM} and OpenCCM 
\cite{MarvieMerle2001} implement this type of component.

In such implementations, each remote component is accessible from any point 
in the network, but 
communication between components in the same address space is very
expensive because method calls still have to go through the CORBA Object
Request Broker (ORB).

There are some techniques for transparently optimizing communication overhead
(e.g. CORBA collocation \cite{ObjectInterconnections18, wang00optimizing}), but 
problems occur when using CORBA services,
because the Portable Object Adapter (POA) is not involved at all.


%==============================================================================
\section{Component adapters}
%==============================================================================
The CCM specification hides most of the CORBA programming details from the 
application developer, but it still forces
the developer to deal with CORBA references, data types and memory management.
While the OMG Java mapping \cite{OMGIDL2Java} is acceptable in this regard,
the C++ mapping \cite{OMGIDL2Cpp} 
does not use the advantages of standard C++ like strings, vectors and lists.

To improve the usability of CCM, we proposed a new approach \cite{teiniker-mkkw:2002} 
that provides 
an easy way to implement business logic without having to pay attention to CORBA details.

\begin{figure}[htbp]
    \begin{center}
        \includegraphics [width=7cm,angle=0] {LCAC_Overview.eps}
        \caption{Local component adapter}
        \label{LcacOverview}    
    \end{center}
\end{figure}

In our approach, we separate the application code from the implementation of the CORBA 
component logic  (Fig.~\ref{LcacOverview}).
For every IDL definition of a CORBA interface or component, we generate a corresponding
interface in the native implementation language.
An adapter class \cite{Gamma95} provides the CORBA mappings, and
links the different implementation of business logic to the
CCM component.

It is important to keep in mind that these adapters do not change the client view
of a component, so our approach still conforms to the CCM standard.




%==============================================================================
\section{Local components}
%==============================================================================

To implement thin components in the same address space we need a specific
component model that defines local components and their interconnections.
In Java there already exists such a thin component model (JavaBeans \cite{Englander1997}),
but in other languages like C++ and Python there are no such specifications.  

So there is a need for a language independent local component model with
mappings to many different programming languages.
Instead of inventing another new component model, we use the CORBA Component
Model in a local manner.
By taking advantage of the adapter concept we can implement local components without a
CORBA shell.
\begin{figure}[htbp]
    \begin{center}
        \includegraphics [width=8cm,angle=0] {Adapter1.eps}
        \caption{Local and remote connection}
        \label{LcacLayerModel}
    \end{center}
\end{figure}

\noindent
As shown in Fig.~\ref{LcacLayerModel}, there is a remote path and a local path
between two component implementations.
When running components in the same address space, there is no need for CORBA
communication. Using the local path for connecting components significantly reduces
the communication overhead.


%==============================================================================
\section{Components with local and remote ports}
%==============================================================================

One of the important issues in {\it Component--Based Software Engineering} (CBSE) 
\cite{CBSE2001} is the granularity of components. 
Fat components increase runtime performance, but their reuse is limited.
On the other hand, thin components lead to significant communication overhead
but are easy to reuse.

In Java, for example, there are two different component models: JavaBeans \cite{Englander1997} 
for thin components and Enterprise JavaBeans (EJB) \cite{EJBSpecificationV2_0} 
for remote components.
Since version 2.0 of EJB there is a local component concept: The whole 
bean is declared as local or remote by extending the {\tt EJBHome, EJBObject} 
interfaces or the {\tt EJBLocalHome, EJBLocalObject} interfaces respectively. 

As with local EJBs, we use local CORBA components for thin components located in 
the same address space to improve performance and reusability.
In contrast to local EJBs, however, we do not have to decide between a local or remote
component because we always implement a local one.
After writing the business logic we can leave some ports local while some other ports
are made remotely accessible by adding a remote adapter. 
Note that the decision between using the local or remote adapters does not
affect the implementation of the business logic; in other words we can scale
the remote accessibility of a component port by port at deployment time.

With this approach we can use the same component model for a wide range of 
component implementations and programming languages.


%==============================================================================
\section{Assemblies of local and remote components}
%==============================================================================

Local components in the CCM specification increase runtime performance, but they
are not accessible from other processes.
To overcome this problem, we build assemblies of local and remote components
using the {\it Session Facade} pattern \cite{Marinescu02}.
\begin{figure}[htbp]
    \begin{center}
        \includegraphics [width=10cm,angle=0] {LCAC_ProcessModel.eps}
        \caption{Component assembly}
        \label{assembly}
    \end{center}
\end{figure}

\noindent
Fig.~\ref{assembly} shows an example of two assemblies running in different
processes.
The information about how to connect an assembly comes from an XML file called the 
{\it assembly descriptor}.
At runtime there is an assembly object within the assembly that creates
instances of components and component connections.

An assembly can be described as a directed graph where the nodes are component
instances and the edges are receptacle to facet connections.
\begin{figure}[htbp]
    \begin{center}
        \includegraphics [width=5cm,angle=0] {AssemblyGraph.eps}
        \includegraphics [width=5cm,angle=0] {AssemblyGraph2.eps}       
        \caption{Assembly instance graphs}
        \label{instanceGraph}
    \end{center}
\end{figure}

If an instance of a session facade component is created, all connected local and
remote components also have to be instantiated and connected.
The assembly object creates an instance of the assembly subgraph as shown in 
Fig.~\ref{instanceGraph}.
In other words, a {\tt create()} call on a session facade component home creates an
assembly instance where only related components are connected  as described in 
the assembly descriptor.
The CORBA reference to the assembly instance is returned by the {\tt create()} call
to the client.

\noindent
From the client's point of view, the session facade component is a fat remote one.
In fact, this fat component is an assembly instance graph consisting of thin remote
and local components that ensure easy reuse of business code.




%==============================================================================
\section{Assemblies of assemblies}
%==============================================================================

When we can use assembly instance graphs in the same way as remote components,
we can then also group them into new assemblies.
These remote assemblies consist of sets of remote session facade components
and their remote connections.
\begin{figure}[htbp]
    \begin{center}
        \includegraphics [width=10cm,angle=0] {RemoteAssembly.eps}
        \caption{Remote assembly}
        \label{remoteAssembly}
    \end{center}
\end{figure}

\noindent
Fig.~\ref{remoteAssembly} shows a remote assembly consisting of three assembly 
instance graphs, each in its own process. 
Note that only the remote components of the instance graphs are visible from
the outside, so the instance graphs look like fat remote CCM components.
This approach is consistent with the CCM specification, but it increases
performance and reusability.











