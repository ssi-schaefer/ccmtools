% $Id$
%==============================================================================
\chapter{CCM Tools Installation}
%==============================================================================


%------------------------------------------------------------------------------
\section{Prerequisites}
%------------------------------------------------------------------------------

To install the CCM Tools, the following programs must be available:
\begin{description}
\item [Java SDK $\ge$ 1.5.x] ({\tt http://java.sun.com/j2se})
\item [Apache Ant $\ge$ 1.6.x] ({\tt http://ant.apache.org})
\item [Python $\ge$ 2.4.x] ({\tt http://python.org})
\item [cpp $\ge$ 3.3.x] ({\tt http://www.gnu.org})
\end{description}

To build the generated C++ components, we also need:
\begin{description}
\item [Confix $\ge$ 1.5.x] ({\tt http://confix.sourceforge.net})
\item [gcc $\ge$ 3.3.x] ({\tt http://www.gnu.org})
\item [mico $\ge$ 2.3.11] ({\tt http://www.mico.org/})
\end{description}


%------------------------------------------------------------------------------
\section{How to get it}
%------------------------------------------------------------------------------

The project is hosted at Sourceforge ({\tt http://ccmtools.sf.net}). See the web
site for releases and announcements.

You can also subscribe to the {\tt ccmtools-announce} mailing list for CCM Tools
release announcements. The {\tt ccmtools-users} mailing list provides a forum
for discussion about using the CCM Tools.


%------------------------------------------------------------------------------
\section{Binary distribution}
%------------------------------------------------------------------------------

Installing the CCM Tools from a binary package is quite simple:
\begin{small}
\begin{verbatim}
$ tar xvzf ccmtools-x.y.z-bin.tar.gz 
\end{verbatim}
\end{small}

\noindent
This package comes with the following structure:
\begin{small}
\begin{verbatim}
ccmtools-x.y.z
|-- bin
|-- lib
`-- templates
    |-- CppLocalTemplates
    |-- CppLocalTestTemplates
    |-- CppRemoteTemplates
    |-- CppRemoteTestTemplates
    |-- IDL2Templates
    |-- IDL3MirrorTemplates
    `-- IDL3Templates
\end{verbatim}
\end{small}

\noindent
Finally, you can set your environment variables:

\begin{small}
\begin{verbatim}
 $ export CCMTOOLS_HOME=<CCM_INSTALL_PATH>
 $ export PATH=$CCMTOOLS_HOME/bin:$PATH	    

 # Additionally, the following settings are needed for using remote
 # components based on the Mico ORB
 $ export CCM_NAME_SERVICE=corbaloc:iiop:1.2@localhost:5050/NameService
 $ export CCM_COMPONENT_REPOSITORY=${CCMTOOLS_HOME} 
 $ export CCM_INSTALL=<MY_INSTALL_PATH>
\end{verbatim}
\end{small}

\noindent
Note that you also need a C++ runtime environment to compile and run the
generated components.
These C++ runtime packages must be installed from source.


\newpage
%------------------------------------------------------------------------------
\section{Source distribution}
%------------------------------------------------------------------------------


%------------------------------------------------------------------------------
\subsection{CCM Tools package:}
%------------------------------------------------------------------------------
Installing the CCM Tools from source requires the following steps:

\begin{small}
\begin{verbatim} 
 $ tar xvzf ccmtools-x.y.z.tar.gz
\end{verbatim}
\end{small}

\noindent
Alternatively, you can check out an up-to-date version from CVS:
\begin{small}
\begin{verbatim}
 $ cvs -d :pserver:anonymous@ccmtools.cvs.sf.net:/cvsroot/ccmtools login
 Password: <press enter>
 $ cvs -d :pserver:anonymous@ccmtools.cvs.sf.net:/cvsroot/ccmtools co ccmtools
\end{verbatim}
\end{small}

\noindent
To build the CCM Tools we use Ant: 
\begin{small}
\begin{verbatim}
 $ cd ccmtools
 $ ant install -Dprefix=<CCM_INSTALL_PATH>
\end{verbatim}
\end{small}

\noindent
Don't forget to set your environment variables properly
(as described in the 'Binary distribution' section). 


%------------------------------------------------------------------------------
\subsection{Java runtime package:}
%------------------------------------------------------------------------------
To access remote CCM components from Java clients, we have to install
a Java client's runtime environment called {\tt java-environment}:

\begin{small}
\begin{verbatim}
 $ tar xvzf java-environment-x.y.z.tar.gz
\end{verbatim}
\end{small}

Alternatively, you can check out an up-to-date version from CVS:
\begin{small}
\begin{verbatim}
 $ cvs -d :pserver:anonymous@ccmtools.cvs.sf.net:/cvsroot/ccmtools \
       co java-environment
\end{verbatim}
\end{small}

To build and install the java-environment we use Ant:
\begin{small}
\begin{verbatim}
 $ cd java-environment
 $ ant install -Dprefix=<CCM_INSTALL_PATH>
\end{verbatim}
\end{small}

\noindent
To used this runtime library from a Java client, 
don't forget to set the {\tt CLASSPATH} variable:
\begin{small}
\begin{verbatim}
 $ export CLASSPATH=<CCM_INSTALL_PATH>/lib/Components.java:$CLASSPATH
\end{verbatim}
\end{small}



%------------------------------------------------------------------------------
\subsection{C++ runtime packages:}
%------------------------------------------------------------------------------
As shown in Fig.~\ref{ccmtools-structure}, to compile and run generated CCM
components, we need a C++ runtime environment.

\noindent
To build and install C++ environment packages as well as generated C++ components, 
we use {\tt Confix}. 
{\tt Confix} is a build tool that is based on {\tt automake} and 
{\tt autoconf} - visit the {\tt confix.sf.net} page to read the
exhaustive manual. 

\noindent
It's a good idea to create a CCM Tools profile in Confix' configuration file
({\tt .confix}), as described in the Confix manual.
\begin{small}
\begin{verbatim}
ccm_tools_profile = {
    'PREFIX': '<MY_INSTALL_PATH>',        # use your own path!
    'BUILDROOT': '<MY_BUILD_PATH>',       # use your own path!
    'ADVANCED': 'true',
    'USE_LIBTOOL': 'true',
    'CONFIX': {
    },
    'CONFIGURE': {
       'ENV': {
          'CC': 'gcc',                     # use your own path!
          'CXX': 'g++',                    # use your own path!    
          'CFLAGS': "-g -O0 -Wall",
          'CXXFLAGS': "-g -O0 -Wall",
          },
       'ARGS': [
        '--with-mico=<MICO_INSTALL_PATH>/lib/mico-setup.sh'
	# use your own mico install path!
       ]
    },
}

PROFILES = {
    'ccmtools': ccm_tools_profile,
    'default' : ccm_tools_profile
}
\end{verbatim}
\end{small}

\noindent
It's important that you substitute your own paths in the {\tt .confix} file.\\
We can configure the {\tt ccm\_tools\_profile} as default profile, thus we 
don't need to use the {\tt --profile=ccmtools} confix option.
Additionally, we advise to set the {\tt ADVANCED} flag to {\tt true} instead of
using the {\tt --advanced} command--line option. 


\noindent
To install the CCM Tools runtime packages, the following steps are needed: 

\begin{small}
\begin{verbatim}
 $ tar xvjf wx-toolsbox-x.y.z.tar.bz2
 $ cd wx-toolsbox-x.y.z
 $ confix.py --bootstrap --configure --make --targets="install"

 $ tar xvjf wx-utils-x.y.z.tar.bz2
 $ cd wx-utils-x.y.z
 $ confix.py --bootstrap --configure --make --targets="install"

 $ tar xvzf cpp-environment-A.B.X.tar.gz
 $ cd cpp-environment
 $ confix.py --packageroot=`pwd`/ccm --bootstrap --configure \
             --make --targets="install"
\end{verbatim}
\end{small}

Note that you can alternatively check out an up-to-date version of the
{\tt cpp-environment} package from CVS:
\begin{small}
\begin{verbatim}
 $ cvs -d :pserver:anonymous@ccmtools.cvs.sf.net:/cvsroot/ccmtools \
       co cpp-environment
\end{verbatim}
\end{small}

Perfect, all tools and libraries have been installed and are ready to work!





