% $Id$
%=============================================================================
\section{Source Files}
%=============================================================================
The IDL specification defines a number of rules for the naming and contents of
IDL source files.

%-----------------------------------------------------------------------------
\subsection{File Naming}
%-----------------------------------------------------------------------------
The names of source files containing IDL definitions must end in {\bf \tt .idl}
(for example, we can define a file named {\tt Components.idl}).

%-----------------------------------------------------------------------------
\subsection{File Format}
%-----------------------------------------------------------------------------
IDL is a free--form language. This means that IDL allows free use of spaces
and newline characters.
Layout and indentation do not carry semantics, so you can choose any textual
style you prefer, but keep in mind that IDL is programming language independent
so don't use language specific prefixes or names.
 

%-----------------------------------------------------------------------------
\subsection{Preprocessing}
%-----------------------------------------------------------------------------
IDL source files are preprocessed. The preprocessor's behavior is identical to
the C++ preprocessor (actually, the CCM Tools use the GNU C preprocessor {\tt 
cpp}).

The most common use of the preprocessor is for {\tt \#include} directives. This
permits an IDL definition to use types defined in a different source file.
You may also want to use the preprocessor to guard against double inclusion of a
file:
\begin{verbatim}
    #ifndef _MYFILENAME_IDL_
    #define _MYFILENAME_IDL_
    
    // some IDL definitions

    #endif /* _MYFILENAME_IDL_ */
\end{verbatim}


%-----------------------------------------------------------------------------
\subsection{Definition Order}
%-----------------------------------------------------------------------------
IDL constructs (modules, interfaces, type definitions) can appear in any order
you prefer.
However, identifiers must be declared befor they can be use.


%-----------------------------------------------------------------------------
\subsection{Comments}
%-----------------------------------------------------------------------------
IDL definitions permit both the C and the C++ style of writing comments:
\begin{verbatim}
    /**
     * This is a legal IDL comment.
     * Note that you can use tools like doxygen to extract
     * comments from IDL files.
     */

    // This comment extends to the end of this line.
\end{verbatim}


%-----------------------------------------------------------------------------
\subsection{Keywords}
%-----------------------------------------------------------------------------
IDL uses a number of keywords, which must be spelled in lowercase (e.g. {\tt
interface}, {\tt struct}, etc.).
There are three exceptions to this lowercase rule: {\tt Object}, {\tt TRUE} and
{\tt FALSE} are all keywords and must be capitalized.

%-----------------------------------------------------------------------------
\subsection{Identifiers}
%-----------------------------------------------------------------------------
Identifiers begin with an alphabetic character followed by any number of
alphabetics, digits, or underscores. Unlike C++ identifiers, IDL identifiers
can't have a leading underscore.

Identifiers are case--insensitive but must be capitalized consistently. 
This rule exists to permit mappings of IDL to languages that ignore case in
identifiers (e.g. Pascal) as well as to languages that treat differently
capitalized identifiers as distinct (e.g. C++, Java).

IDL permits you to create identifiers that happen to be keywords in one or more
implementation languages, but to make life easier, you should try to avoid IDL
identifiers that are likely to be implementation language keywords.

\newpage
