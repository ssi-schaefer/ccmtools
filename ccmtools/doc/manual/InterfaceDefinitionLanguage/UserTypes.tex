%=============================================================================
\section{User--Defined IDL Types}
%=============================================================================
In addition to providing the build--in basic types, IDL permits you to define
complex types: enumerations, structures and sequences. You can also use {\tt 
typedef} to explicitly name a type.
\vspace{5mm}


%-----------------------------------------------------------------------------
\subsection{Named Types}
%-----------------------------------------------------------------------------

You can use {\tt typedef} to create a new name for a type or to rename an
existing type.

\vspace{2mm}
Example:
\begin{verbatim}
    module world
    {
        typedef long TimeStamp;
    }; // end of module world
\end{verbatim}

Be careful about the semantics of IDL {\tt typedef}. It depends on the language
mapping whether an IDL {\tt typedef} results in a new, separate type or only an
alias. 
To avoid potential problems, you should define each logical type exactly once
and then use that definition consistently throughout your specification.


%-----------------------------------------------------------------------------
\subsection{Enumerations}
%-----------------------------------------------------------------------------
An IDL enumerated type definition looks much like the C++ version.

\vspace{2mm}
Example:
\begin{verbatim}
    module world
    {
        enum Color 
        {
            red, 
            green, 
            blue
        };
    }; // end of module world
\end{verbatim}

This example introduces a type named {\tt Color} that becomes a new type in its
own right - there is no need to use a {\tt typedef} to name the type.


%-----------------------------------------------------------------------------
\subsection{Structures}
%-----------------------------------------------------------------------------
IDL supports structures containing one or more named members of arbitrary type,
including user--defined complex types.

\vspace{4mm}
Example:
\begin{verbatim}
    module world
    {
        struct TimeOfDay
        {
            short hh;
            short mm;
            short ss;
        };
    }; // end of module world
\end{verbatim}

This definition introduces a new type called {\tt TimeOfDay}.
Structure definition form a namespace, so the names of the structure members
need to be unique only within their enclosing structure.

%-----------------------------------------------------------------------------
\subsection{Sequences}
%-----------------------------------------------------------------------------
Sequences are variable--length vectors that can contain any element type. 

\vspace{2mm}
Example:
\begin{verbatim}
    module world
    {
        typedef sequence<Color> Colors; 
    }; // end of module world
\end{verbatim}

A sequence can hold any number of elements up to the memory limits of your
platform. 

\vspace{5mm}

