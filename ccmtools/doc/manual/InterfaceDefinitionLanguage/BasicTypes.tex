%=============================================================================
\section{Basic IDL Types}
%=============================================================================
IDL provides a number of build--in basic types. 
The CORBA specification requires that language mappings preserve the {\it size}
of basic IDL types.
To avoid restricting the possible target environments and languages, the 
specification leaves the size and range requirements for IDL basic types loose.


%-----------------------------------------------------------------------------
\subsection{Integer Types}
%-----------------------------------------------------------------------------

\begin{itemize}
  \item {\tt short} (range from $-2^{15}$ to $2^{15}-1$, size $\geq$ 16 bits)
  \item {\tt long} (range from $-2^{31}$ to $2^{31}-1$, size $\geq$ 32 bits)
  \item {\tt unsigned short} (range from $0$ to $2^{16}-1$, size  $\geq$
  16 bits)
  \item {\tt unsigned long} (range from $0$ to $2^{32}-1$, size $\geq$ 32
  bits)
\end{itemize}

%-----------------------------------------------------------------------------
\subsection{Floating--Point Types}
%-----------------------------------------------------------------------------

\begin{itemize}
  \item {\tt float} (IEEE single--precision, size $\geq$ 32 bits)
  \item {\tt double} (IEEE double--precision, size $\geq$ 64 bits)
\end{itemize}


%-----------------------------------------------------------------------------
\subsection{Characters}
%-----------------------------------------------------------------------------
\begin{itemize}
  \item {\tt char} (ISO Latin--1, $\geq$ 8 bits)
  \item {\tt wchar} ($\geq$ 16 bits)
\end{itemize}

%-----------------------------------------------------------------------------
\subsection{Strings}
%-----------------------------------------------------------------------------
\begin{itemize}
  \item {\tt string} (ISO Latin--1, variable--length)
  \item {\tt wstring} (variable--length)
\end{itemize}

%-----------------------------------------------------------------------------
\subsection{Booleans}
%-----------------------------------------------------------------------------
Boolean values can have only the values {\tt TRUE} and {\tt FALSE}. 

%-----------------------------------------------------------------------------
\subsection{Octets}
%-----------------------------------------------------------------------------
The IDL type {\tt octet} is an 8--bit type that is guaranteed not to undergo any
changes in representation as it is transmitted between processes.

%-----------------------------------------------------------------------------
\subsection{Type any}
%-----------------------------------------------------------------------------
Type {\tt any} is a universal container type. A value of type {\tt any} can hold
a value of any other IDL type (e.g. {\tt long}, {\tt string}, or even another
value of type {\tt any}).
Type {\tt any} is useful when you don't know at compile time what IDL types you
will eventually need to transmit between client and server, you can find out at
runtime what type of value is contained in the {\tt any}.
It is recommended to use a {\tt typedef} construct to introduce any types in your
interface definition files. 

\vspace{2mm}
Example:
\begin{verbatim}
    typedef any GenericType;
\end{verbatim}
  
