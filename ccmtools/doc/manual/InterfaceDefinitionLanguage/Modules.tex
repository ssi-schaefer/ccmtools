%=============================================================================
\section{Modules}
%=============================================================================
IDL uses the {\tt module} construct to create namespaces.
Modules combine related definitions into a logical group and prevent pollution
of the global namespace.
Identifiers in a module need be unique only within that module.
The IDL parser searches for the definition of an identifier from the 
innermost scope outward toward the outermost scope.

\vspace{2mm}
Example:
\begin{verbatim}
    module world
    {
        /** Some IDL definitions */
    };
\end{verbatim}

In addition, modules can contain other modules, so you can create nested
hierarchies. 

\newpage
Example:
\begin{verbatim}
    module world
    {
        /** Some IDL definitions */
         
         module europe
         {
             /** Other IDL definitions */             
         };
    };
\end{verbatim}

Modules can be reopened. 
Incremental definition of modules is useful if specifications are written by a
number of developers (instead of creating a giant definition inside a single
module, you can break the module into a number of separate source files).

\vspace{2mm}
Example:
\begin{verbatim}
    module world
    {
        /** Some IDL definitions */
    };
    
    // ...
    
    module world
    {     
        /** Other IDL definitions */             
    };
\end{verbatim}

The CCM Tools don't support global scope IDL definitions, thus, every IDL artefact 
must be placed within at least one module.

