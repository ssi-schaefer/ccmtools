% $Id$
%==============================================================================
\section{Use Case 5: Mirror Component Concept}
\label{section:MirrorComponentConcept}
%==============================================================================

A serious problem in component testing is to satisfy all component ports without
having other components to connect to.
In a software system, a component always is connected to other components via
facets and receptacles.
To develop and test a single component, the CCM Tools support the generation of 
IDL mirror component definitions. 

\vspace{3mm}
A {\bf Mirror Component} represents the complement of a given component. For every
facet in the original component there is a receptacle in the mirror component
and vice versa.
Within the mirror component, a developer can implement test cases which
describes the desired behavior of a component.

\vspace{3mm}
Implementation of local mirror component tests require the
following activities:
\begin{itemize}
	\item Model the component's structure in IDL (see section~\ref{section:ComponentDefinition}). 
	\item Implement the local component (see section~\ref{section:LocalC++ComponentImplementation})

	\item Generate the mirror component definition.
	\item Generate the local mirror component logic.
	\item Implement mirror component test cases.

\end{itemize}

In the following sections, we will implement mirror component test cases for a
local C++ component implementation.


%------------------------------------------------------------------------------
\subsection{Generate the mirror component definition}
\label{subsection:GenerateMirrorComponentDefinition}
%------------------------------------------------------------------------------

To create an IDL mirror component definition from the {\tt Server} component we
use the following CCM Tools call: 
\begin{footnotesize}
\begin{verbatim}
> cd Login

> ccmidl -idl3mirror                                \
         -Iidl3repo/interface -Iidl3repo/component  \
         -o idl3repo                                \
         idl3repo/component/application/Server*.idl
\end{verbatim}
\end{footnotesize}

Now, the IDL repository directory contains the new mirror component definition:
\begin{footnotesize}
\begin{verbatim}
Login
|-- idl3repo
|   |-- component
|   |   `-- application
|   |       |-- Server.idl
|   |       |-- ServerHome.idl
|   |       |-- ServerHomeMirror.idl
|   |       `-- ServerMirror.idl
\end{verbatim}
\end{footnotesize}

These new files are:
\begin{itemize}
  \item {\tt ServerMirror.idl} \\
  This file contains the mirror component's IDL definitions:
    \begin{footnotesize}
    \begin{lstlisting}[language=C++]
    #include <application/Login.idl>

    module application {
        component ServerMirror
        {
            uses ::application::Login login;
        };
    }; // /module application
	\end{lstlisting} 
    \end{footnotesize}
  \item {\tt ServerHomeMirror.idl} \\
    This file contains the IDL definition of the mirror component's home:
    \begin{footnotesize}
    \begin{lstlisting}[language=C++]
    #include <application/ServerMirror.idl>

    module application {
        home ServerHomeMirror
            manages ::application::ServerMirror
        {
        };
    }; // /module application
	\end{lstlisting} 
    \end{footnotesize}
\end{itemize}

Based on these new IDL definitions, we can use existing CCM Tools generators to
realize this mirror component.


%------------------------------------------------------------------------------
\subsection{Generate the local mirror component logic}
\label{subsection:GenerateLocalMirrorComponentLogicCxx}
%------------------------------------------------------------------------------
A mirror component can be generated in the same way as a component under test.
In addition to the mirror component logic, the {\tt c++local-test} generator creates
a {\tt \_check\_*} file which handles the test setup where both components will be
instantiated and connected by their facets and receptacles.   

\begin{footnotesize}
\begin{verbatim}
> cd c++/server

> ccmtools c++local                                                  \
           -I../../idl3repo/interface -I../../idl3repo/component     \
           -a                                                        \
           -o src/component/ServerMirror                             \
           ../../idl3repo/component/application/ServerMirror.idl     \
           ../../idl3repo/component/application/ServerHomeMirror.idl

> ccmtools c++local-test                                             \
           -I../../idl3repo/interface -I../../idl3repo/component     \
           -o src/component/ServerMirror                             \
           ../../idl3repo/component/application/Server.idl           
\end{verbatim}
\end{footnotesize}

These generated mirror component files are stored parallel to the other 
component directory:
\begin{footnotesize}
\begin{verbatim}
Login/c++/server
`-- src
    |-- component
    |   |-- Server
    |   `-- ServerMirror
    |       |-- CCM_application_ccm_local
    |       |-- CCM_application_ccm_local_share
    |       |-- ServerHomeMirror_impl.cc
    |       |-- ServerHomeMirror_impl.h
    |       |-- ServerMirror_impl.cc
    |       |-- ServerMirror_impl.h
    |       |-- application_ccm_local_ServerHomeMirror_entry.h
    |       `-- test
    |           `-- _check_application_ccm_local_Server.cc
\end{verbatim}
\end{footnotesize}



%------------------------------------------------------------------------------
\subsection{Implement mirror component test cases}
\label{subsection:ImplementMirrorComponentTestCasesCxx}
%------------------------------------------------------------------------------
As the mirror component's business logic, we implement two test cases.
We used the {\tt -a} option to force the CCM Tools to generate application class
skeletons.
In the {\tt ServerMirror\_impl.cc} class we implement the following code snippet:
\begin{footnotesize}
\begin{lstlisting}[language=C++]
void
ServerMirror_impl::ccm_activate()
    throw(::Components::ccm::local::CCMException)
{
    try
    {
        SmartPtr<Login> login =  ctx->get_connection_login();
        try 
        {
            PersonData person;
            person.id = 277;
            person.name = "eteinik";
            person.password = "eteinik";
            person.group = USER;
            bool result = login->isValidUser(person);
            if(result) 
            {
                cout << "Welcome " << person.name << endl;
            }
            else 
            {
                cout << "We don't know you !!!" << endl;
            }
        }
        catch(InvalidPersonData& e) 
        {
            cout << "Error: InvalidPersonData!!" << endl;
        }
        
        try 
        {
            PersonData person;
            person.id = 0;
            person.name = "";
            person.password = "";
            person.group = USER;      
            login->isValidUser(person);
            assert(false);
        }
        catch(InvalidPersonData& e) 
        {
            cout << "OK, caught InvalidPersonData exception!" << endl;
        }
    }
    catch(::Components::ccm::local::Exception& e)
    {
        cerr << "ERROR: " << e.what() << endl;
    }
}
\end{lstlisting} 
\end{footnotesize}

{\tt ccm\_activate()} is a callback method which will be called from the component 
logic during the client's {\tt configuration\_complete()} call.
Within this callback method, we implement these test cases. 
To call operations on the original component's facet, we use {\tt get\_connection\_login()} 
in the mirror component test case to get the connected receptacle reference. 

\vspace{3mm}
Finally, we use Confix to build both components and to run the unit test:
\begin{footnotesize}
\begin{verbatim}
> ccmconfix -confix2 -o src -pname "login" -pversion "1.0.0"

> confix2.py --packageroot=`pwd`/src --bootstrap --configure --make --targets=check
\end{verbatim}
\end{footnotesize}
Not surprisingly, you should see the following output on your command line:
\begin{footnotesize}
\begin{verbatim}
Welcome eteinik
OK, caught InvalidPersonData exception!
PASS: login__check_application_ccm_local_Server
==================
All 1 tests passed
==================
\end{verbatim}
\end{footnotesize}

The mirror component test concept handles the complete test setup for you, thus,
you can focus on your test cases only.

\newpage

