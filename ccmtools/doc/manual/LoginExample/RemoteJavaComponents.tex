% $Id$
%==============================================================================
\section{Use Case 4: Remote Java Components}
\label{section:RemoteJavaComponentImplementation}
%==============================================================================

In this CCM Tools use case, we implement a remote Java component and a local
test client which uses a Java client library component.
This use case is adequate for developers who implement large and distributed
Java applications.
 
\vspace{3mm}
The implementation of remote Java components requires the following activities:
\begin{itemize}
	\item Model the component's structure in IDL (see section~\ref{section:ComponentDefinition}). 
	\item Implement the local component (see section~\ref{section:LocalJavaComponentImplementation}).
	\item Generate the remote component logic.
	\item Implement a minimal CORBA server.
	\item Generate the client library component.
	\item Implement a local component client.
\end{itemize}

You will see in the next sections that most of the development steps needed to
extend local components to remote Java components can be completely automated by
the CCM Tools.

\vspace{3mm}
This section assumes that there is already a local Java component implementation
which can be transform into a remote one.


%------------------------------------------------------------------------------
\subsection{Generate remote component logic in Java}
\label{subsection:GenerateRemoteComponentLogicJava}
%------------------------------------------------------------------------------
We use CORBA middleware to overcome process boundaries.
For noncritical applications you can use the Java build--in CORBA ORB.
To turn a local Java component into a remote component, we have to generate CORBA
stubs and skeletons as well as a bunch of adapter and converter classes.

\vspace{3mm}
Generation of CORBA stubs and skeletons includes an IDL3 to IDL2 transformation
and multiple calls to an external IDL compiler. We can use the CCM Tools script
{\tt ccmtools-idl} to call Java's build--in IDL compiler:
\begin{footnotesize}
\begin{verbatim}
> ccmidl -idl2                                                 \
         -I../../idl3repo/interface -I../../idl3repo/component \ 
         -o ./src-gen/idl2                                     \
         ../../idl3repo/interface/application/*.idl

> ccmidl -idl2                                                 \ 
         -I../../idl3repo/interface -I../../idl3repo/component \
         -o ./src-gen/idl2                                     \
         ../../idl3repo/component/application/Server*.idl

> ccmtools-idl -java                                    \
               -I${CCMTOOLS_HOME}/idl -I./src-gen/idl2  \ 
               -o ./src-gen                             \
               ./src-gen/idl2/*.idl
\end{verbatim}
\end{footnotesize}

Another pair of CCM Tools calls create CORBA adapter and converter classes:
\begin{footnotesize}
\begin{verbatim}
> ccmjava -remote                                               \
          -I../../idl3repo/interface -I../../idl3repo/component \
          -o ./src-gen                                          \
          ../../idl3repo/interface/application/*.idl 

> ccmjava -remote                                               \
          -I../../idl3repo/interface -I../../idl3repo/component \ 
          -o ./src-gen                                          \
          ../../idl3repo/component/application/Server*.idl
\end{verbatim}
\end{footnotesize}

After all, you can see the following directory structure: 
\begin{footnotesize}
\begin{verbatim}
Login/java/server
`-- src-gen
    |-- application
    |   `-- ccm
    |       |-- local
    |       `-- remote
    `-- idl2
\end{verbatim}
\end{footnotesize}

Each single file in this {\tt src-gen} directory has been generated, so, 
there is no need for a CVS like check--in.


%------------------------------------------------------------------------------
\subsection{Implement minimal CORBA server in Java}
\label{subsection:ImplementMinimalCorbaServerJava}
%------------------------------------------------------------------------------
For starting up a remote component, a minimal CORBA server can be implemented 
which initialize the ORB (by passing command line parameters), deploys the remote
component home and runs the ORB.

\begin{footnotesize}
\begin{lstlisting}[language=Java]
import org.omg.CORBA.ORB;
import ccm.local.ServiceLocator;

public class Server
{
    public static void main(String[] args)
    {
        try
        {
            // Set up the ServiceLocator singleton
            ORB orb = ORB.init(args, null);
            ServiceLocator.instance().setCorbaOrb(orb);

            application.ccm.remote.ServerHomeDeployment.deploy("ServerHome");
            System.out.println("ServerHome server is running...");
            orb.run();
        }
        catch (Exception e)
        {
            e.printStackTrace();
        }
    }
\end{lstlisting}
\end{footnotesize}

This server class called {\tt Server} is stored in the {\tt src} directory:
\begin{footnotesize}
\begin{verbatim}
Login/java/server
|-- src
|   |-- Server.java
|   `-- application
\end{verbatim}
\end{footnotesize}

Don't forget to start a CORBA name service, because a component deployment
implies the registration of the component home object.
\begin{footnotesize}
\begin{verbatim}
> orbd -ORBInitialPort 5050
\end{verbatim}
\end{footnotesize}

We can run the same Ant build script as we have used for building the
local Java component. 
\begin{footnotesize}
\begin{verbatim}
> ant
\end{verbatim}
\end{footnotesize}

Finally, we can start the minimal CORBA server to activate our remote component:
\begin{footnotesize}
\begin{verbatim}
> java -enableassertions                               \
       -cp $CCMTOOLS_HOME/lib/ccm-runtime.jar:./build  \
       Server                                          \
       -ORBInitRef NameService=corbaloc:iiop:1.2@localhost:5050/NameService
\end{verbatim}
\end{footnotesize}

Your console output should look like:
\begin{footnotesize}
\begin{verbatim}
ServerHome server is running...
\end{verbatim}
\end{footnotesize}


%------------------------------------------------------------------------------
\subsection{Generate a client library component in Java}
\label{subsection:GenerateClientLibComponentInJava}
%------------------------------------------------------------------------------

We have seen that a local component can be extended to a remote component
without any business logic changes.
To bring the same advantage to the client side, CCM Tools support so called
{\bf Client Library Components} which are a local proxies for remote components.
Client library components implement the same interfaces as local components and
delegate each local call to the corresponding remote components.
 
\vspace{3mm}
Remote clients need CORBA stubs and skeletons of the used IDL interfaces.
As a matter of course, this redundant step can be skipped if you develop both
remote component and client on the same box.
\begin{footnotesize}
\begin{verbatim}
> ccmidl -idl2                                                 \
         -I../../idl3repo/interface -I../../idl3repo/component \ 
         -o ./src-gen/idl2                                     \
         ../../idl3repo/interface/application/*.idl

> ccmidl -idl2                                                 \ 
         -I../../idl3repo/interface -I../../idl3repo/component \
         -o ./src-gen/idl2                                     \
         ../../idl3repo/component/application/Server*.idl

> ccmtools-idl -java                                   \
               -I${CCMTOOLS_HOME}/idl -I./src-gen/idl2 \ 
               -o ./src-gen                            \
               ./src-gen/idl2/*.idl
\end{verbatim}
\end{footnotesize}

On top of CORBA stubs and skeletons we generate the client library component:
\begin{footnotesize}
\begin{verbatim}
> ccmjava -iface -clientlib                                     \
          -I../../idl3repo/interface -I../../idl3repo/component \
          -o ./src-gen                                          \
          ../../idl3repo/interface/application/*.idl

> ccmjava -iface -clientlib                                     \
          -I../../idl3repo/interface -I../../idl3repo/component \
          -o ./src-gen                                          \
          ../../idl3repo/component/application/Server*.idl
\end{verbatim}
\end{footnotesize}

All generated files are collected in the temporary {\tt src-gen} directory.



%------------------------------------------------------------------------------
\subsection{Implement local component client in Java}
\label{subsection:ImplementRemoteComponentClientInJava}
%------------------------------------------------------------------------------
Based on the generated client library component, we can implement a
component client much like a simple local Java component client.

\vspace{3mm}
As shown in the following listing, only the client's setup and tear down
sections are different to a local component client implementation:
\begin{footnotesize}
\begin{lstlisting}[language=Java]
import org.omg.CORBA.ORB;
import application.ccm.local.*;
import Components.ccm.local.HomeFinder;
import ccm.local.ServiceLocator;

public class Client
{
    public static void main(String[] args)
    {
        try
        {
            ORB orb = ORB.init(args, null);
            ServiceLocator.instance().setCorbaOrb(orb);
            ServerHomeClientLibDeployment.deploy("ServerHome");
        }
        catch (Exception e)
        {
            e.printStackTrace();
        }

        // TODO: client's business logic implementation 
        //       (see collocated client implementation)
        
        try
        {
            ServerHomeClientLibDeployment.undeploy("ServerHome");
        }
        catch (Exception e)
        {
            e.printStackTrace();
        }
    }
}
\end{lstlisting}
\end{footnotesize}

We store the client class in a {\tt src} directory:
\begin{footnotesize}
\begin{verbatim}
Login/java/client
|-- src
|   `-- Client.java
`-- src-gen
\end{verbatim}
\end{footnotesize}

As Ant build script we can reuse the script from the server--side.
\begin{footnotesize}
\begin{verbatim}
> ant
\end{verbatim}
\end{footnotesize}

Make sure that the remote component is running before you start the client with:
\begin{footnotesize}
\begin{verbatim}
> java -enableassertions                               \
       -cp $CCMTOOLS_HOME/lib/ccm-runtime.jar:./build  \
       Client                                          \
       -ORBInitRef NameService=corbaloc:iiop:1.2@localhost:5050/NameService
\end{verbatim}
\end{footnotesize}

\begin{footnotesize}
\begin{verbatim}
Welcome eteinik
OK, caught InvalidPersonData exception!
\end{verbatim}
\end{footnotesize}
Congratulations, the remote Java component is working now!

\newpage
