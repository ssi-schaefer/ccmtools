% $Id$
%==============================================================================
\section{IDL Repository Directory}
\label{section:IdlRepositoryDirectory}
%==============================================================================

First of all, we store the IDL source code in a file called {\tt Login.idl}
and tell the CCM Tools to generate an {\bf IDL Repository Directory}.
This {\tt idl3repo} directory contains all defined IDL artefacts in separated
files and in a uniform structure.

\begin{verbatim}
	> ccmidl -idl3 -o ./idl3repo Login.idl
\end{verbatim}

After this step, we should have the following directory structure:
\begin{verbatim}
        |-- Login.idl
        |-- idl3repo
        |   |-- component
        |   |   `-- application
        |   |       |-- Server.idl
        |   |       `-- ServerHome.idl
        |   `-- interface
        |       `-- application
        |           |-- Group.idl
        |           |-- InvalidPersonData.idl
        |           |-- Login.idl
        |           `-- PersonData.idl
\end{verbatim}

In the repository, which is the starting point for all other CCM Tools activities,
there are two subdirectories:
\begin{itemize}
 	\item The {\tt interface} directory contains all IDL interface,
 	parameter, exception, etc. definitions.
 	  
  	\item The {\tt component} directory contains all component and
  	home definitions.
\end{itemize}

Each IDL artefact is stored in its own file within a directory that
conforms to the defined IDL module hierarchy.
For example, the interface {\tt Login} has been defined in the module 
{\tt application}, thus, this interface is stored in the directory 
{\tt interface/application} in a file named {\tt Login.idl} within the IDL 
repository.

\vspace{3mm}
From the developers point of view, it does not matter if the component
definitions are stored in a single or in multiple source files, the generated 
{\tt idl3repo} directory tree is the same in both cases.

 \newpage
