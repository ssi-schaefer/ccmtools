% $Id$
%==============================================================================
\section{Use Case 1: Local C++ Components}
\label{section:LocalC++ComponentImplementation}
%==============================================================================

To introduce the first CCM Tools use case, we implement a local C++ component and a
collocated unit test. This use case is adequate for a developers who implements
large but modular C++ applications.

\vspace{3mm}
The implementation of local C++ components requires the following activities:
\begin{itemize}
	\item Model a component's structure in IDL 
			(see section~\ref{section:ComponentDefinition}). 
	\item Generate component logic (= component skeletons).
	\item Implement a component's business logic.
	\item Implement a component's client.

\end{itemize}

It is an important point that the modeling of IDL interfaces and components 
is completely independent of component implementations.
As you will see, we use the IDL artifacts stored in the IDL repository directory
to generate both C++ and Java implementations.  
 
 
%------------------------------------------------------------------------------
\subsection{Generate component logic}
\label{subsection:GenerateComponentLogic}
%------------------------------------------------------------------------------

Let's start development. From the IDL repository directory the CCM Tools
generate a component skeleton which establishes the component's structure,
provides C++ interfaces to clients or other components and uses the C++ runtime 
environment.

\begin{verbatim}
> mkdir c++
> mkdir c++/server
> cd c++/server

> ccmtools c++local -I../../idl3repo/interface -I../../idl3repo/component \
                    -o ./src/interface                                    \
                    ../../idl3repo/interface/application/*.idl            

> ccmtools c++local -I../../idl3repo/interface -I../../idl3repo/component \
                    -a -o ./src/component/Server                          \
                    ../../idl3repo/component/application/Server*.idl
\end{verbatim}

After this code generation step, the following directory structure has been created:
\begin{verbatim}
Login/c++/server
  `-- src
    |-- component
    |   `-- Server
    |       |-- CCM_application_ccm_local_component_Server
    |       |-- CCM_application_ccm_local_component_Server_share
    |       `-- application_ccm_local_component_Server_ServerHome_entry.h
    `-- interface
        |-- CCM_application_ccm_local
        `-- CCM_application_ccm_local_adapter
\end{verbatim}

Basically, all directories starting with '{\tt CCM\_}' contain component logic
which is completely generated (so there is no need to check--in these directories
into a CVS like system).

\vspace{3mm}
The component logic fills the gap between a component's interfaces and its
business logic implementation. 

\vspace{3mm}
Note that generated component logic can change between different CCM Tools
versions to improve a component's non functional behavior. Such changes do neither
affect component interfaces nor your business logic implementation which
realizes the functional behavior of components.   


%------------------------------------------------------------------------------
\subsection{Implement business logic}
\label{subsection:ImplementC++BusinessLogic}
%------------------------------------------------------------------------------

Component business logic must be embedded in the generated component logic.
To make life easier, we used the {\tt -a} option during code generation.
This flag forces the code generator to generate application skeletons.

\vspace{3mm}
You can find these application skeletons in the {\tt src/component/Server}
subdirectory:

\begin{verbatim}
Login/c++/server
  `-- src
    |-- component
    |   `-- Server
    |       |-- ServerHome_impl.cc
    |       |-- ServerHome_impl.h
    |       |-- Server_impl.cc
    |       |-- Server_impl.h
    |       |-- Server_login_impl.cc
    |       |-- Server_login_impl.h
\end{verbatim}

As a developer, you are responsible for these files because they represent the
component's business logic (you should check--in these files into a CVS like system).

\vspace{3mm}
There is a direct relationship between IDL and these business logic files:
\begin{itemize}
	\item {\tt ServerHome\_impl.*}\\
	For each component home, an implementation class is generated which provides an
	implementation of the default {\tt create()} operation.
	Additionally, the {\tt ServerHome\_impl.cc} file contains the implementation of
	the global:
	\begin{verbatim}
    create_application_ccm_local_component_Server_ServerHome()
    \end{verbatim}
 	function which represents the business logic entry point used by the generated 
 	component logic. 

\item {\tt Server\_impl.*}\\
	For each component, an implementation class is generated which provides 
	default implementations of the component's callback operations.
	
\item {\tt Server\_login\_impl.*}\\
	For each facet, an implementation class is generated which provides operation
	skeletons for hosting business logic implementations.
\end{itemize}

It is a good idea to generate these application skeletons only once - when starting 
component implementation. 
Small changes in IDL definitions can be appended pretty easy to these
implementation classes manually.

 \vspace{3mm}
Note that these implementation files are not overwritten by the CCM Tools.
The generator replaces only untouched source files, otherwise the new
generated files are stored with a {\tt .new} suffix.

 \vspace{3mm}
To implement the Login example's business logic, you open the 
{\tt Server\_login\_impl.cc} file and implement the following code snippet:

\begin{footnotesize}
\begin{lstlisting}[language=C++]
bool
login_impl::isValidUser(
    const application::ccm::local::PersonData& person)
    throw(::ccm::local::Components::CCMException,
          application::ccm::local::InvalidPersonData )
{
    if(person.name.length() == 0)
        throw application::ccm::local::InvalidPersonData();

    if(person.id == 277 
       && person.name == "eteinik"
       && person.group == USER) 
    {
        return true;
    }
    else 
    {
        return false;
    }
}
\end{lstlisting}
\end{footnotesize}

Now, we can use Confix to build our component example. To tell Confix which
directory should be built, a {\tt Makefile.py} file must be created in each
source code directory.
You can delegate this work to CCM Tools:
\begin{verbatim}
> ccmconfix -makefiles -o ./src -pname "login" -pversion "1.0.0"
\end{verbatim}
 
Finally, you start Confix to build all generated and manually implemented source
files:
\begin{verbatim}
> confix.py --packageroot=`pwd`/src --bootstrap --configure --make 
\end{verbatim}

Hey, we a now ready to test this local C++ component implementation.


%------------------------------------------------------------------------------
\subsection{Implement a local component client in C++}
\label{subsection:ImplementLocalComponentClient}
%------------------------------------------------------------------------------

Instead of a real client with a complex GUI, we simply implement a unit test for
the component we have built in the last section.

\vspace{3mm}
We create a {\tt src/component/Server/test} directory and store the following
code in a file called {\tt \_check\_local\_component\_Server.cc}:
 
\begin{footnotesize} 
\begin{lstlisting}[language=C++]
#include <cassert>
#include <iostream>

#include <WX/Utils/debug.h>
#include <WX/Utils/smartptr.h>

#include <ccm/local/Components/CCM.h>
#include <ccm/local/HomeFinder.h>

#include <application/ccm/local/component/Server/Server_gen.h>
#include <application/ccm/local/component/Server/ServerHome_gen.h>

using namespace std;
using namespace WX::Utils;
using namespace ccm::local;
using namespace application::ccm::local;

int main(int argc, char *argv[])
{
  SmartPtr<component::Server::Server> server;
  SmartPtr<Login> login;
  Components::HomeFinder* homeFinder = HomeFinder::Instance();
  if(deploy_application_ccm_local_component_Server_ServerHome("ServerHome")) 
  {
      cerr << "ERROR: Can't deploy component homes!" << endl;
      return -1;
  }

  try 
  {
      SmartPtr<component::Server::ServerHome> serverHome(
        dynamic_cast<component::Server::ServerHome*>
          (homeFinder->find_home_by_name("ServerHome").ptr()));

      server = serverHome->create();
      login = server->provide_login();
      server->configuration_complete();

      // Implement your test cases here !!!
    
      server->remove();
  } 
  catch(Components::Exception& e ) 
  {
      cout << "ERROR: " << e.what() << endl;
      return -2;
  } 

  if(undeploy_application_ccm_local_component_Server_ServerHome("ServerHome")) 
  {
    cerr << "ERROR: Can't undeploy component home!" << endl;
    return -3;
  }
}
\end{lstlisting}
\end{footnotesize}

This code snipped is very similar in all unit tests of such simple components.
It deploys the component home object, creates a component instance, uses 
the component's equivalent interface to get a facet, and completes the
configuration phase. 
Following the setup process, we can execute our component test cases (we will
discuss the implementation of these test case later).
Finally, we remove the component instance and undeploy the component home object.
Each functional test case can be inserted into this unit test template shown above.


\vspace{3mm}
Our first test shows the usage of the {\tt Server} component and its {\tt login}
facet. We fill the {\tt PersonData} structure with valid data and call the
{\tt isValidUser()} operation. Depending on the component's result we print out
a message to the console.

\begin{footnotesize}
\begin{lstlisting}[language=C++]
    try 
    {
        PersonData person;
        person.id = 277;
        person.name = "eteinik";
        person.password = "eteinik";
        person.group = USER;
	
        bool result = login->isValidUser(person);
        if(result) 
        {
            cout << "Welcome " << person.name << endl;
        }
        else 
        {
            cout << "Sorry, we don't know you !!!" << endl;
        }
    }
    catch(InvalidPersonData& e) 
    {
        cout << "Error: InvalidPersonData!!" << endl;
    }
\end{lstlisting}
\end{footnotesize}


The second test shows the component's behavior for an invalid {\tt PersonData}
structure. The test expects an {\tt InvalidPersonData} exception to succeed.

\begin{footnotesize}
\begin{lstlisting}[language=C++]
    try 
    {
        PersonData person;
        person.id = 0;
        person.name = "";
        person.password = "";
        person.group = USER;

        login->isValidUser(person);
        assert(false);
    }
    catch(InvalidPersonData& e) 
    {
        cout << "OK, caught InvalidPersonData exception!" << endl;
    }
\end{lstlisting}
\end{footnotesize}

It is up to you to decide if you put both test cases into the same {\tt \_check\_*}
file or to implement each test case in its own file.
Note that each {\tt \_check\_*} file will end in a separate executable, thus,
for huge applications you will need a lot of disk space.

To run these unit tests, we use Confix again:
\begin{verbatim}
> touch src/component/Server/test/Makefile.py
> confix.py --packageroot=`pwd`/src --bootstrap --configure \
            --make --targets=check 
\end{verbatim}

At the end of this build process, you hopefully see an output like:
\begin{verbatim}
Welcome eteinik
OK, caught InvalidPersonData exception!
PASS: login__check_local_component_Server
==================
All 1 tests passed
==================
\end{verbatim}

Of course, to implement a component for such a simple functionality is somewhat
academical, but this example shows how simple a component development cycle can
be by using CCM Tools. 


\newpage
