%******************************************************************************
\chapter{Introduction}
\label{chapter:introduction}
%******************************************************************************

The CCM Tools are a set of programs and libraries used to generate, package and
deploy CORBA components. The tools implement parts of the CCM specification
\cite{CCMSpecification} with some extensions to improve the usability and
performance of the component model. The {\it CCM Tools User's Guide\/} provides
instructions for using the CCM Tools. This document, the {\it CCM Tools
Developer's Manual\/}, is aimed at developers and provides details on the inner
workings of the CCM Tools.

\section{Component tools}

\begin{figure}[!htb]
\centering \includegraphics[width=14cm]{ToolsOverview}
\caption{Parts of the CCM Tools package.}
\label{fig:intro-ToolsOverview}
\end{figure}

To help the component developer, the CCM Tools provide the tools shown in
Figure~\ref{fig:intro-ToolsOverview}. These tools include the parts described
below.

\paragraph{CCM Metamodel Library}

The CCM specification defines a metamodel for the IDL3 syntax. This metamodel is
a set of classes that are capable of representing the syntactic elements in
IDL3. We implemented a metamodel library in Java that allows creation of, and
navigation through, CCM models. Using this library we can clearly separate the
parser and the code generators, using a CCM model as the communication between
the two parts. The parser creates a model object for every part of the source
code matching an IDL grammar rule, and the generator tools read the model to
generate code. Chapter~\ref{chapter:ccm-metamodel-library} describes the CCM
Tools' construction of the CCM metamodel library.

\paragraph{IDL3 Parser}

The IDL3 Parser reads an IDL3 file, checks the syntax of the IDL source code and
creates a CCM model, using the CCM metamodel library, in memory. This CCM model
is the starting point for all code generator tools.
Chapter~\ref{chapter:idl3-parser} describes the IDL3 parser in detail.

\paragraph{Local C++ Component Generator}

The CCM Tools include a generator tool that creates the local component logic
from a given CCM model. The generated code contains C++ implementations of a
component that can run in the same address space (hence called a `local
component'). This implementation includes operations for providing, using,
connecting, and disconnecting components and their facets. These generated
operations are referred to as {\bf component logic}. After generating the
component logic, the component developer needs to write the {\bf business
logic}; that is, the functional implementation of the component so that it
becomes useful in a component based software system. See
Chapter~\ref{chapter:component-generator-tools} for more information about the
component generator library.

\paragraph{Mirror Component Generators}

After building a component it is a good idea to run a set of functional tests on
it. Such testing can provide helpful component specific debugging and prevent
large, difficult to locate, system bugs later in development.

To provide a test environment for every component, the CCM Tools can create a
mirror component (a facet becomes a receptacle and vice versa) described in a
mirror IDL3 file. We use the C++ component generators with the mirror IDL3 files
to generate the code for mirror components. Then we use a C++ mirror test
generator to create a basic test executable that connects each component with
its mirror and calls the functions available for each component.

\paragraph{IDL2 Generator}

To implement CORBA components, IDL3 source code needs to be reduced to IDL2,
which can be interpreted by classic IDL compilers. The transformation from IDL3
to IDL2 also adds some operations needed for navigation between components and
their ports (equivalent operations). The CCM Tools support this transformation
using an IDL2 generator tool that creates IDL2 code based on a CCM model.

\paragraph{Remote C++ Component Generator}

Note that the local components can only be used in a common address space and
must be implemented in the same programming language. To overcome these
limitations, the CCM Tools have a remote component generator that produces C++
code to interface the local components with CORBA. The remote component logic is
thus a superset of the local component logic.

\paragraph{Component Descriptor Generator}

To describe the component for the deployment and assembling processes, the OMG
defines a {\it CORBA Component Descriptor} (CCD) file. This is an XML file
containing a short description of a component and its ports. We map the CCM
model to an XML--DOM tree that can be written as an XML file. The CCD file is
also used by the code generators to get some additional information about the
components being generated (version, vendor, etc.).

\paragraph{UML Parser}

Optionally, the description of the component's interfaces can be contained in a
UML diagram. We need a UML parser that reads the UML--XMI file, builds a UML
model (based on the UML metamodel defined by the OMG) and transposes the model
to a standard IDL3 file. The CCM Tools can use the NSUML UML metamodel library
from NovoSoft, but the OMG has not yet defined an UML profile for the CCM. Thus
a UML tool will likely be on the back burner for some time.

\paragraph{Component Packaging Tool}

After generating and writing the component logic and the component descriptor
file, we have to package these files into a zip file called, predictably, a
component package. The component packaging tool provides these functionalities
to the component developer.

\paragraph{Component Deployment Tool}

On the target host the component package must be unzipped and the component must
be deployed in the application server. The component deployment tool provides
these functionalities to the component deployer.

\section{Assembly tools}

After creating the component logic and writing the business logic, we can use
the component from a simple client program. A big advantage of the CORBA
component model is the ability to connect components together to form larger
structures called {\bf component assemblies}.

A component assembly is a set of components (with their component descriptors)
and a {\it Component Assembly Descriptor} (CAD). Based on the CAD we can
generate an assembly object that instantiates components and connects them
together to form a component assembly.

\begin{figure}[!htb]
\includegraphics[width=14cm]{AssemblyTools}
\caption{CCM component assembly tools.}
\label{fig:intro-AssemblyTools}
\end{figure}

Figure~\ref{fig:intro-AssemblyTools} provides a diagram of the assembly tools in
the CCM Tools. These tools include:

\paragraph{Assembly Descriptor Generator}

The component descriptor files are the base for the higher--level assembly
descriptor, which describes the components of the assembly and their
connections. That means that all information of the assembly descriptor comes
from the component descriptors of the related components (and some additional
data from a GUI).

\paragraph{Assembly Object Generator}

At runtime a managing object is needed that can establish an assembly instance.
The assembly object creates the component instances and connects their
receptacles and facets. All information for generating an assembly object comes
from the assembly descriptor (or its DOM model in memory). Note that this object
must eventually be able to create local or local/remote assembly instances.

\paragraph{UML Parser}

As with components, there should be a way to define component assemblies in a
UML diagram. Therefore we need a UML parser that reads the UML--XMI file and
translates the data into the DOM model used by the assembly descriptor. The OMG
has not yet defined a mapping between UML and CCM assemblies.

\paragraph{Assembly Packaging Tool}

After generating the component assembly and its descriptor file we have to
package these files into a zip file called a component assembly package. The
assembly packaging tool provides these functionalities to the assembly
developer.

\paragraph{Assembly Deployment Tool}

On the target host the component assembly package must be unzipped and the
assembly must be deployed in the application server. The assembly deployment
tool provides these functionalities to the assembly deployer.

\section{Testing}

Another important issue is {\bf testing}. We have to test our applications on
different levels during development, as listed below:

\paragraph{Class Level Testing}

Every class of the business logic that will be part of a component has to be
tested.

\paragraph{Component Level Testing}

For every component we create a counter component that looks like a mirror to
the original component. This counter component has an receptacle for every facet
of the original component and vice versa. Connecting a component with its mirror
component with a simple test client allows component developers to quickly
isolate component level errors in the business code.

\paragraph{Assembly Level Testing}

After testing each component, we have to be sure that a set of connected
components, the component assembly, also works correctly.

Of course, there must be tool support for testing on these different levels of
development to make this job more efficient.

