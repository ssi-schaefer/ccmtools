%$Id$
%==============================================================================
\section{Introduction}
%==============================================================================

Software components are executable units of independent production, acquisition,
and deployment that can be composed into a functioning system.
Components are for composition.
Composition enables prefabricated components to be reused by rearranging them 
in new composites.
Traditional software development can broadly be devided into two approaches:
\begin{itemize}
\item At one extreme, a project is developed entirely from scratch, with the
help of only programming tools and libraries.

\item At the other extreme, everything is build of standard software which
is bought and parametrized to provide a solution that is close enough to what
is needed.
\end{itemize}

\noindent 
The concept of component software represents a middle path that could solve 
this problem.
Although each bought component is a standardized product, the process of
component assembly allows the opportunity for significant customization.
{\it Component--Based Software Engineering} (CBSE) 
\cite{Szyperski02,IVICA2002,CBSE2001}
has become recognized as a new subdiscipline of software engineering. 
The major goals of CBSE are:
\begin{itemize}
\item To provide support for development of software systems as assemblies of
	components.
\item To support development of software components as reusable entities.
\item To facilitate the maintenance and upgrade of systems by customizing
   	and replacing their components.
\end{itemize}

\noindent
Software components were initially considered to be analogous to hardware
components in general and to integrated circuits (IC) in particular.
But software technology is an engineering discipline in its own right, with its
own principles and laws. Therefore, such analogies break down quickly when 
going into technical details.



