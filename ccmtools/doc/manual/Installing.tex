%******************************************************************************
\chapter{Installing the CCM Tools}
%******************************************************************************

Installing the CCM Tools generally follows the standard GNU install procedure.
That is, the CCM Tools package comes with a configure script and a series of
{\tt Makefile.in} files which allow for on--site configuration and
customization. If you are already familiar with this installation process, you
might just want to unpack the source tarball and run {\tt ./configure --help} to
get information about specific configuration parameters. Then {\tt ./configure},
{\tt make}, {\tt make install}, and update your {\tt CLASSPATH} and {\tt PATH}
environment variables appropriately.

If you are not familiar with this process, or if you need clarification on a
particular step, the rest of this chapter describes the unpack, configure,
build, install, and environment update processes in more detail.

%==============================================================================
\section{Unpacking, Configuring, Building, and Installing}
%==============================================================================

Once you have the CCM Tools source tarball, locate a suitable directory for
unpacking it. Common places to do this (on Linux) are {\tt \$HOME/src}, or, if
you decide to install as the superuser, in {\tt /usr/local/src} or {\tt
/usr/src}. For the rest of this section, as an example, I will assume you want
to build in a directory under {\tt /home/leif/src}.

First, open up a terminal window, and change directories to your chosen
location.
\begin{verbatim}
leif@fridge:~ $ cd src
\end{verbatim}
Then unpack the source tarball, and change directories into the newly-created
ccmtools-A.B.C directory (where A.B.C is the version of CCM Tools that you are
using).
\begin{verbatim}
leif@fridge:~/src $ tar zxf ccmtools-A.B.C.tar.gz
leif@fridge:~/src $ cd ccmtools-A.B.C
\end{verbatim}

The standard GNU install process involves two major steps: configuring and
building. To help with platform--independence for installing and such, the
tarball comes with a configure script. To invoke it, just type {\tt
./configure}. You can get a list of available options by giving the {\tt --help}
option.

If you want to install the CCM Tools into a nonstandard location (the default is
system dependent), specify the {\tt --prefix=DIR} option. If you would like to
build the tests and documentation by default, specify the {\tt --enable-tests}
and {\tt --enable-docs} options, respectively. Note that enabling tests while
building the CCM Tools is not the same as building test or mirror components;
you can keep the CCM Tools tests disabled at this point and still be able to
generate test components later.

\begin{verbatim}
leif@fridge:~/src $ ./configure --help
`configure' configures ccmtools A.B.C to adapt to many ...

Usage: ./configure [OPTION]... [VAR=VALUE]...
:
:
leif@fridge:~/src $ ./configure --prefix=/home/leif/sandbox
checking for a BSD-compatible install... /usr/bin/install -c
checking whether build environment is sane... yes
checking for gawk... gawk
checking whether make sets $(MAKE)... yes
:
:
\end{verbatim}

Then, to build the package, simply type {\tt make}. Install the package with
{\tt make install}. Note that normally the package needs to be installed to
function correctly, as the code generators otherwise have no way of finding
their template sets.

%==============================================================================
\section{Updating Environment Variables}
%==============================================================================

After you install the CCM Tools package, you will probably need to update a few
environment variables. In particular, two Java archive (jar) files need to be
included in your {\tt CLASSPATH} environment variable: {\tt antlr.jar}, and {\tt
ccmtools.jar}. These three files are installed in the {\tt share/java/}
directory under your install prefix. So, for example, if you configured the CCM
Tools package with {\tt --prefix=/a/b}, you will need to add {\tt
/a/b/share/java/antlr.jar} and {\tt /a/b/share/java/ccmtools.jar} to your {\tt
CLASSPATH}. If you did not give an install prefix to the configure script, the
package will be installed under a system-dependent default directory.
